% Subsection '3c'
% Testing of IPv6-only (on behalf of Raul Lopes - Brunel)
%
An IPv6-only WLCG production cluster, composed of an ARC-CE head node,
two worker nodes and three (SQUID-based) web cache nodes based has been
in operation at Brunel University since March 2018. Given the value
we place on early detection of IPv4-only code sections (especially
non-address-translatable constructs such as the use of IPv4 literals in
data structures and signaling - see above, \S\ref{ssec:ipv6onlymove}),
no transition techniques (e.g. NAT64/DNS64) were used for this
infrastructure.
\par
WLCG production jobs for three (out of four) major LHC experiments were routed
to this IPv6-only cluster, with LHCB jobs running successfully in
2018, CMS jobs (submitted via a dedicated queue) running successfully in 2019,
and ATLAS jobs, also handled by a special IPv6-only queue, requiring an
in-depth, and still partly on-going, investigation of issues mainly within the
Frontier\footnote{\tt http://frontier.cern.ch/} distributed database
service.
\par
This reality check does confirm that IPv4 is still {\it required} in part
of the WLCG software base, with services failing in case IPv4 connectivity
cannot be established. While the development time that has been spent in 
early troubleshooting and linting of these cases will definitely be rewarded
as the transition progresses, we plan to complement this study with 
an assessment on how many of the residual issues aren't or cannot be covered
by available address-translation techniques.

