%section 1


 


The HEPiX IPv6 Working Group \cite{ipv6wg} has been investigating the many issues related to the 
move of the Worldwide LHC Computing Grid (WLCG) services to dual-stack IPv6/IPv4 networking, thus enabling the use of
IPv6-only CPU resources as agreed by the WLCG Management Board and presented by us at 
CHEP2018 \cite{ipv6chep2018}. 

The dual-stack deployment does however result in a networking environment
which is more complex than when using just IPv6. Some WLCG services, e.g.
the EOS storage system at CERN \cite{eos}, are already using IPv6-only for internal communication,
where possible. Several Broadband/Mobile-phone companies, such as T-Mobile in the USA and
BT/EE in the UK, now use IPv6-only networking with connectivity to the IPv4
legacy world enabled by the use of NAT64 (RFC6146\cite{rfc}), DNS64 (RFC6147\cite{rfc}) and 464XLAT (RFC6877\cite{rfc}). Large companies, such
as Facebook, use IPv6-only networking within their internal networks, there
being good management and performance reasons for this. Based on these examples
of IPv6-only networking, we have therefore been motivated to investigate the future removal of the IPv4 protocol in
places within the WLCG infrastructure.

This paper presents the status of the WLCG transition to dual-stack services, together with 
our work and plans for moving to an IPv6-only networking environment for WLCG.

