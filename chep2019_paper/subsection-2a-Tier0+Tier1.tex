
% Subsection '2a'
% Tier 0 and Tier 1's
%
%\subsection{Deployment at the Tier-0 and Tier-1 sites}
\subsection{Deployment at Tier-0 and Tier-1's}
After the aforementioned ten years the storage environment is almost completely dual-stack ready and at CERN the Tier-0 and the 14 Tier-1s IPv6 and IPv4 dual-stack is nearly fully enabled. Only the Tier-1 site the Kurchatov Institute in Moscow, part of the Russian Federation, is still running on ipv4 only. This enables a total of 96\% ipv6 accessible storage of LHC as shown in table~\ref{tab:t012stor}.
\begin{table}[h]
\centering
\caption{Fraction of Tier-1 and Tier-2 storage available over IPv6}
\label{tab:t012stor}
\begin{tabular}{lccccc}
\hline
& ALICE & ATLAS & CMS & LHCb & Global \\\hline
Tier-1 storage & 78\% & 96\% & 100\% & 94\% & 96\% \\
Tier-2 storage & 86\% & 59\% &  89\% & 75\% & 74\% \\\hline
\end{tabular}
\end{table}
The FTS server at FNAL is still running in IPv4 preferred mode. There was a long standing malfunctioned transfer issue to IPv4 only US-Tier-2 sites which is solved now. This last server will get deployed in dual-stack as soon as possible.
