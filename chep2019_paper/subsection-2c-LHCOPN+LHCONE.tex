\subsection{LHCOPN and LHCONE}
The LHCOPN and LHCONE are both virtual private networks (VPN) serving the Large Hadron Collider Experiments. Since the end of 2016 both networks are dual-stack ready. LHCOPN is a CERN (Tier-0) centric star network mainly deployed for the distribution of the raw detector data to the Tier-1 sites. Even though the majority of Tier-1 sites are dual-stack ready and the protocol IPv6 is the one preferred, we still have the situation that IPv6 is the only transfer protocol. However, a tendency towards IPv6 file transfers is  recognizable. The LHCONE network consists of appr. 140 sites connected through Virtual Routing and Forwarding (VRF) implementations at 26 different network service providers (NSP). All connected end sites deploying a Border Gateway Protocol (BGP) routing table and advertising their own Classless Inter-Domain Routing (CIDR) to the connecting NSP. The network itself has been IPv6 ready since quite a long time. The connected end sites are becoming more and more IPv6 ready. This is recognizable at the  transfer protocol changes from IPv4 towards IPv6. The high usage of the IPv4 transfer protocol visualizes that the fraction of IPv4 only sites is still quite substantial.
