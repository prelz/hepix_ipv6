\subsection{LHCOPN and LHCONE}

The LHCOPN and LHCONE are both virtual private networks (VPN) serving the Large Hadron Colider Experiments. Both networks are from the end of 2016 anward dual-stack ready. LHCOPN is a CERN (Tier-0) centric star network mainley deployed for the distribuion of the raw detector data to the tier-1 sites. Since the majority of Tier-1 sites are dual-stack ready and even while the protocol IPv6 is prefered it is still not the situation that IPv6 is the only transfers protocol, but a tendence towards IPv6 fle transfers are recognizible. LHCONE is a netwrork of close to 140 sites connected trough Virtual Routing and Forwarding implementations at 26 diffrent network service providers (NSP). All connected endsites deploying a Border Gateway Protocol (BGP) routing table and adverticing their own CIDR to the connecting NSP. The network itself is already since long IPv6 ready. The connected end sites are becomming more and more IPv6 ready. This is recogizable at the  transfer protocol changes from IPv4 towards IPv6. Out of the high usage of the IPv4 transfer protocol it is still recogisable the the fraction of IPv4 only sites is quite substancial.
