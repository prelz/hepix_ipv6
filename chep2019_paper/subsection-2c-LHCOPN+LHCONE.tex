\subsection{LHCOPN and LHCONE}
The LHCOPN and LHCONE are both virtual private networks (VPN) serving the Large Hadron Collider Experiments. Since the end of 2016 both networks are dual-stack ready. LHCOPN is a CERN (Tier-0) centric star network mainly deployed for the distribution of the raw detector data to the Tier-1 sites. Even though the majority of Tier-1 sites are dual-stack ready and the protocol IPv6 is the one preferred, we still have the situation that IPv6 is not the only transfer protocol, in part because of the FTS server at Fermilab still running in IPv4-mode. The LHCONE network consists of approximately 140 sites connected through Virtual Routing and Forwarding (VRF) implementations at 26 different network service providers (NSP). All connected end sites deploying a Border Gateway Protocol (BGP) routing table and advertising their own Classless Inter-Domain Routing (CIDR) to the connecting NSP. The network itself has been IPv6-ready since quite a long time. The connected end sites are becoming more and more IPv6-ready. The evidence for the transition is visible by the data transfer protocol changing from IPv4 to IPv6. The still significant usage of the IPv4 transfer protocol demonstrates that the fraction of IPv4-only sites is still quite substantial. 
