\subsection{LHCOPN and LHCONE}

The LHCOPN and LHCONE are both virtual private networks (VPN) serving the Large Hadron Colider Experiments. Both networks are from the end of 2016 onward dual-stack ready. LHCOPN is a CERN (Tier-0) centric star network mainly deployed for the distribution of the raw detector data to the Tier-1 sites. Since the majority of Tier-1 sites are dual-stack ready and even while the protocol IPv6 is preferred it is still not the situation that IPv6 is the only transfer protocol, but a tendency towards IPv6 file transfers are recognizable. LHCONE is a network of close to 140 sites connected trough Virtual Routing and Forwarding implementations at 26 different network service providers (NSP). All connected end sites deploying a Border Gateway Protocol (BGP) routing table and advertising their own Classless Inter-Domain Routing (CIDR) to the connecting NSP. The network itself is already since long IPv6 ready. The connected end sites are becoming more and more IPv6 ready. This is recognizable at the  transfer protocol changes from IPv4 towards IPv6. The high usage of the IPv4 transfer protocol visualizes that the fraction of IPv4 only sites is still quite substantial.
