% section 3 IPv6-only networking
% sub-section a) Aims of moving to IPv6-only and issues to be tackled  (Francesco)
% sub-section b) The case for an IPv6-only LHCOPN  (Edoardo)
% sub-section c) Testing of IPv6-only (Raul Lopes - Brunel)
% sub-section d) Future plans (Francesco) - moved to section 4 conclusions

A few years ago, RFC 6586 (\cite{rfc}) reported on a survey on IPv6-only
networking for mainstream applications (gaming, telephony,
multimedia, etc.) and observed that {\it \guillemotleft it is possible to employ IPv6-only
networking\guillemotright} and that {\it \guillemotleft for large classes of applications
there are no downsides or the downsides are negligible\guillemotright}.
This, along with the good working relations we established over
the past years with the HEP software stack developers,
encouraged us to test 
scenarios where the transient complication of running and managing
two independent network stacks is eventually over and we are `back' to just
running IPv6.

\subsection{Aims of moving to IPv6-only and issues to be tackled}
\label{ssec:ipv6onlymove}
A dual-stack IPv6/IPv4 setup includes many components and services that need
to be deployed twice {\it and kept in sync}: firewall rules and access
lists, address assignment services, routing rules, network monitoring,
diagnostics and
intrusion detection infrastructures - just to name a few.
Removing this duplication is highly desirable both for better maintainability
and cost-saving.
However, this {\it requires} a technical
solution to access any site and service that may remain accessible via IPv4
{\it only}.
This trailing remainder of sites will be hopefully
shrinking but will likely exist for a very long time (see
e.g. \cite{ipv6trans} and references therein).
It is actually expected that after large blocks of public IPv4
addresses start being returned to the market, their market price will
decrease and offset any economic drive connected to the IPv4 address shortage,
{\it relieving} the pressure for migration.\par

The standard solution for accessing IPV4-only services from an IPv6-only
network is the deployment of DNS64 (RFC6147) and stateful
NAT64 (RFC6146 \cite{rfc}) services. DNS64 maps names that are resolved to
an IPv4 address only ('{\tt A}' records) to a synthesized IPv6 address composed
of a default prefix\footnote{Usually {\tt 64:ff9b::}.} plus the four bytes of
the IPv4 address. Traffic towards the DNS64 prefix is then routed to
a NAT64 service attached to the public IPv4 network. This will in turn map the
source ports, translate the IP packets to IPv4 and convert any return traffic
back to IPv6. While NAT64 and DNS64 services start being incorporated by
several technology developers (especially in the 'carrier-grade' NAT 
appliance market),
just a few open-source reference implementations exist for UNIX, with
JOOL\footnote{\tt https://www.jool.mx} apparently being the only one 
under active development.\par

While IPv6-only environments can present a few operational challenges that
can be worked around\footnote{Some OS-specific network management tools,
firewall appliances and network monitoring and diagnostic tools were found
to be defective or immature, see RFC 6586 \cite{rfc} for details.},
one-step (6$\rightarrow$4) address translation techniques do and will fail whenever 
IPv4 literal addresses are explicitely handled, stored or signaled by
network applications or protocols. We feel that the time is ripe to
start identifying this class of applications and protocols and direct an early effort
at cleaning them of {\it any} usage or reference to IPv4 literals. While 
two-step (4$\rightarrow$6$\rightarrow$4) address translation techniques
such as XLAT464\footnote{See RFC 6877 \cite{rfc}. XLAT464 keeps a private
IPv4 address assigned to devices connected to IPv6-only networks and
performs an additional address translation at the device level.}
are currently being added to 
network stacks especially at the request of telephony carriers that operate
IPv6-only networks, we see this extra indirection as an (inefficient)
workaround that just hides issues that should be fixed at the application
level. Locating this issues as early as possible motivates the experimental
operation of typical
WLCG sites with IPv6-only networking, as described
later (\S \ref{ssec:ipv6onlytest}).

\subsection{The case for an IPv6-only LHCOPN}
\input subsection-3b-lhcopn.tex

\subsection{Testing of IPv6-only}
\label{ssec:ipv6onlytest}
\input subsection-3c-ipv6only-testing.tex

