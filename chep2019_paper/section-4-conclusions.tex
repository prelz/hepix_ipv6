% section 4 Conclusions (Dave)
%\subsection{Future plans}

We have presented the status of the WLCG transition to the use of dual-stack IPv6/IPv4. The Tier-1 transition is nearly complete and 
more than 70\% of the Tier-2 storage is available over IPv6. The transition will only be completed once we remove the complexity of
dual-stack networking and the WLCG core infrastructure is IPv6-only.


Insufficiently tested or immature code and the requirement that IPv6-based
tools and infrastructures perform at least equally well as their IPv4
counterparts have been the opposite, conflicting poles of every IPv6
deployment effort so far. This continues to be true in the process
of completing the WLCG transition. We conclude that
testing activities, and the consequent early detection of further application
development needs, will keep the working group busy. We plan
to increase the number of sites and stakeholders involved in testing IPv6-only 
scenarios. The aim is to stress-test existing networking software components
that implement any needed transition protocol (especially NAT64 and DNS64, as 
their implementations under current maintenance are rare) and detect
residual uses of IPv4 literals or IPv4-specific APIs in both applications and
network protocols as early as possible.\par
Any use of IPv4 that cannot respond properly to a NAT64-mediated
transaction\footnote{More complex and inefficient address
translation solutions such as the deployment of `customer'-side address
translation for RFC 6877 \cite{rfc} 464XLAT or RFC 7597/9 MAP-E/T  should be
seen as options of last resort, see \S \ref{ssec:ipv6onlymove} above.}
should be seen as an issue to be reported, tracked and
addressed by developers: we plan to deal with these just as we did
with the lack of IPv6 support or the incorrect address selection
strategies we were able to identify so far.\par
Once we are confident that IPv6-only scenarios work well and that
all issues found with the use of transition protocols have been fixed, we will
propose a timetable for the deployment of an IPv6-only networking 
environment for WLCG.
