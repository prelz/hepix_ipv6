\documentclass{webofc}
\usepackage[varg]{txfonts}   % Web of Conferences font
%
% Put here some packages required or/and some personnal commands
%
%
\begin{document}
%
\title{IPv6-only networking on WLCG}
%
% subtitle is optionnal
%
%%%\subtitle{Do you have a subtitle?\\ If so, write it here}

\author{
  \firstname{Marian} \lastname{Babik}\inst{1}\and
  \firstname{Martin} \lastname{Bly}\inst{2}\and
  \firstname{Tim} \lastname{Chown}\inst{3}\and
  \firstname{Ji\v{r}i} \lastname{Chudoba}\inst{4}\and
  \firstname{Catalin} \lastname{Condurache}\inst{2}\and
  \firstname{Thomas} \lastname{Finnern}\inst{5}\and
  \firstname{Terry} \lastname{Froy}\inst{6}\and
  \firstname{Costin} \lastname{Grigoras}\inst{1}\and
  \firstname{Kashif} \lastname{Hafeez}\inst{2}\and
  \firstname{Bruno Heinrich} \lastname{Hoeft}\inst{7}\and
  \firstname{David P.} \lastname{Kelsey}\inst{2}\thanks{\email{david.kelsey@stfc.ac.uk}} \and
  \firstname{Fernando} \lastname{L\'opez~Mu\~noz}\inst{8,9}\fnsep\and
  \firstname{Edoardo} \lastname{Martelli}\inst{1}\and
  \firstname{Raja} \lastname{Nandakumar}\inst{2}\and
  \firstname{Kars} \lastname{Ohrenberg}\inst{5}\and
  \firstname{Francesco} \lastname{Prelz}\inst{10}\and
  \firstname{Duncan} \lastname{Rand}\inst{11}\and
  \firstname{Andrea} \lastname{Sciab\`a}\inst{1}
}

\institute{ 
  European Organization for Nuclear Research (CERN), CH-1211 Geneva 23, Switzerland
\and
  STFC Rutherford Appleton Laboratory, Harwell Campus, Didcot, Oxfordshire OX11 0QX, United Kingdom
\and
  JISC, Lumen House, Library Avenue, Harwell Campus, Didcot, Oxfordshire OX11 0SG, United Kingdom
\and
  Institute of Physics, Academy of Sciences of the Czech Republic, Na Slovance 2 182 21 Prague 8, Czech Republic
\and
  Deutsches Elektronen-Synchrotron DESY, Notkestra\ss e 85, D-22607 Hamburg, Germany
\and
  Queen Mary University of London, Mile End Road, London E1 4NS, United Kingdom
\and
  Karlsruher Institut f\"ur Technologie, Hermann-von-Helmholtz-Platz 1, D-76344 Eggenstein-Leopoldshafen, Germany 
\and
  Port d'Informaci\'o Cient\'ifica, Campus UAB, Edifici D, E-08193 Bellaterra, Spain
\and
  Centro de Investigaciones Energ\'eticas, Medioambientales y Tecnol\'ogicas (CIEMAT), Madrid, Spain
\and
  INFN, Sezione di Milano, via G. Celoria 16, I-20133 Milano, Italy
\and
  Imperial College London, South Kensington Campus, London SW7 2AZ, United Kingdom
          }

\abstract{%


The use of IPv6 on the general internet continues to grow.
Several Broadband/Mobile-phone companies, such as T-Mobile in the USA and
BT/EE in the UK, now use IPv6-only networking with connectivity to the IPv4
legacy world enabled by the use of NAT64/DNS64/464XLAT. Large companies, such
as Facebook, use IPv6-only networking within their internal networks, there
being good management and performance reasons for this. The transition of
WLCG central and storage services to dual-stack IPv4/IPv6 is progressing well,
thus enabling the use of IPv6-only CPU resources as agreed by the WLCG
Management Board and presented by us at earlier CHEP conferences.

During the last year, the HEPiX IPv6 working group has not only been chasing
and supporting the transition to dual-stack services, but has also been
encouraging network monitoring providers to allow for filtering of plots by
the IP protocol used. We have investigated and fixed the reasons for the use
of IPv4 between two dual-stack endpoints when IPv6 should be preferred.
We present this work and the tests that have been made of IPv6-only CPU
showing the successful use of IPv6 protocols in accessing WLCG services.

The dual-stack deployment does however result in a networking environment
which is much more complex than when using just IPv6. Some services, e.g.
the EOS storage system at CERN, are using IPv6-only for internal communication,
where possible. The group is investigating the removal of the IPv4 protocol in
more places. We will present the areas where this could be useful and possible
and suggest a timetable for being able to turn off IPv4 in this way.
}
%
\maketitle
%
\section{Introduction}
\label{sec-intro}
The world's Regional Internet Registries are rapidly running out of available IPv4 addresses and the 
general slow transition to IPv6 continues. The Worldwide Large Hadron Collider Grid (WLCG) and the LHC experiments 
may soon have access to worker nodes or virtual machines possessing only an IPv6-routable address. The HEPiX
IPv6 Working Group [Ref 1] has been investigating the many issues feeding into the move to the use of IPv6 in HEP and WLCG.
The group's paper at CHEP2013 [Ref 2] described the aims of the group and the testing of dual-stack IPv6/IPv4 
services that had been completed at that point. In the last 18 months the group has worked more closely with the
4 major LHC experiments and identified the main use case for the support of IPv6-only clients on WLCG. The groups
activities, including testing of dual-stack data storage services, during the last 18 months are presented in this 
paper together with its future plans.

% Etc. etc. etc.

%For one-column wide figures use syntax of figure~\ref{fig-1}
%\begin{figure}[h]
%% Use the relevant command for your figure-insertion program
%% to insert the figure file.
%\centering
%\includegraphics[width=1cm,clip]{tiger}
%\caption{Please write your figure caption here}
%\label{fig-1}       % Give a unique label
%\end{figure}
%
%For two-column wide figures use syntax of figure~\ref{fig-2}
%\begin{figure*}
%\centering
%% Use the relevant command for your figure-insertion program
%% to insert the figure file. See example above.
%% If not, use
%\vspace*{5cm}       % Give the correct figure height in cm
%\caption{Please write your figure caption here}
%\label{fig-2}       % Give a unique label
%\end{figure*}
%
%For figure with sidecaption legend use syntax of figure
%\begin{figure}
%% Use the relevant command for your figure-insertion program
%% to insert the figure file.
%\centering
%\sidecaption
%\includegraphics[width=5cm,clip]{tiger}
%\caption{Please write your figure caption here}
%\label{fig-3}       % Give a unique label
%\end{figure}
%
%For tables use syntax in table~\ref{tab-1}.
%\begin{table}
%\centering
%\caption{Please write your table caption here}
%\label{tab-1}       % Give a unique label
%% For LaTeX tables you can use
%\begin{tabular}{lll}
%\hline
%first & second & third  \\\hline
%number & number & number \\
%number & number & number \\\hline
%\end{tabular}
%% Or use
%\vspace*{5cm}  % with the correct table height
%\end{table}
%
% BibTeX or Biber users please use (the style is already called in the class, ensure that the "woc.bst" style is in your local directory)
% \bibliography{name or your bibliography database}
%
% Non-BibTeX users please use
%
%\begin{thebibliography}{}
%
% and use \bibitem to create references.
%
%\bibitem{rfc} All Internet Engineering Task Force Requests For Comments (RFC) do
%cuments are available
%from URLs such as http://www.ietf.org/rfc/rfcNNNN.txt where NNNN is the RFC numb
%er, for example {\tt http://www.ietf.org/rfc/rfc2460.txt}
% Format for Journal Reference
%Journal Author, Journal \textbf{Volume}, page numbers (year)
% Format for books
%\bibitem{RefB}
%Book Author, \textit{Book title} (Publisher, place, year) page numbers
% etc
%\end{thebibliography}

\begin{thebibliography}{1}
%
% and use \bibitem to create references.
%
% Format for Journal Reference
%Journal Author, Journal \textbf{Volume}, page numbers (year)
% Format for books
%\bibitem{RefB}
%Book Author, \textit{Book title} (Publisher, place, year) page numbers
% etc


%section 1 references
\bibitem{ipv6wg} The HEPiX IPv6 Working Group web site is to be found at {\tt http://hepix-ipv6.web.cern.ch}

\bibitem{ipv6chep2016} 
Babik M et al 2016 Deployment of IPv6-only CPU resources at WLCG sites {\it J. Phys.: Conf. Ser. {\bf898} 082033}

%section 2 references
%section 2 Tier 0/1

%section 2 Tier 2

%section 2 Experiments
\bibitem{alien}  http://cds.cern.ch/record/1171677?ln=en

\bibitem{jalien} http://cds.cern.ch/record/2026281?ln=en

\bibitem{glideinwms} 
http://iopscience.iop.org/article/10.1088/1742-6596/119/6/062044

\bibitem{htcondor}
Douglas Thain, Todd Tannenbaum, and Miron Livny.
\newblock Distributed computing in practice: the condor experience.
\newblock {\em Concurrency - Practice and Experience}, 17(2-4):323--356, 2005.

\bibitem{dirac} Dirac :  A Tsaregorodtsev and the Dirac Project 2014 J. Phys.: Conf.
Ser. 513 032096
LHCb : LHCb collaboration, A. A. Alves Jr. et al., The LHCb detector at
the LHC, JINST 3 (2008) S08005

%section 3 references
\bibitem{sam}
A~Aimar et al 
%A~Aguado Corman, P~Andrade, S~Belov, J~Delgado Fernandez, B~Garrido Bear, M~Georgiou, E~Karavakis, L~Magnoni, R~Rama Ballesteros, H~Riahi, J~Rodriguez Martinez, P~Saiz, and D~Zolnai.
\newblock Unified monitoring architecture for it and grid services.
\newblock {\em Journal of Physics: Conference Series}, 898(9):092033, 2017.

\bibitem{etf}
Marian Babik.
\newblock Experiments {Test} {Framework} ({ETF}).

\bibitem{perfsonar} 
Andreas Hanemann, Jeff~W. Boote, Eric~L. Boyd, J{\'e}r{\^o}me Durand, Loukik
  Kudarimoti, Roman {\L}apacz, D.~Martin Swany, Szymon Trocha, and Jason
  Zurawski.
\newblock Perfsonar: A service oriented architecture for multi-domain network
  monitoring.
\newblock In Boualem Benatallah, Fabio Casati, and Paolo Traverso, editors,
  {\em Service-Oriented Computing - ICSOC 2005}, pages 241--254, Berlin,
  Heidelberg, 2005. Springer Berlin Heidelberg.

\bibitem{wlcg-NTWG}
S.~McKee, M.~Babik et al 
%S.~Campana, A.~Di Girolamo, T.~Wildish, J.~Closier, S.~Roiser, C.~Grigoras, I.~Vukotic, M.~Salichos, Kaushik De, V.~Garonne, J.A.D. Cruz, A.~Forti, C.J. Walker, D.~Rand, A.~de~Salvo, E.~Mazzoni, I.~Gable, F.~Chollet, L.~Caillat, F.~Schaer, Hsin-Yen Chen, U.~Tigerstedt, G.~Duckeck, B.~Hoeft, A.~Petzold, F.~Lopez, J.~Flix, S.~Stancu, J.~Shade, M.~O'Connor, V.~Kotlyar, and J.~Zurawski.
\newblock Integrating network and transfer metrics to optimize transfer
  efficiency and experiment workflows.
\newblock {\em Journal of Physics: Conference Series}, 664(5):052003, 2015.

\bibitem{psmad}  
perfSONAR Consortium.
\newblock {perfSONAR} {Monitoring and Debugging Dashboard (MADDASH)}.

\bibitem{grafana-ipv6}
CERN~{MONIT} team.
\newblock {CERN} {MONIT} {Grafana} {Dashboard}.

\bibitem{638647551}
A~A Ayllon, M~Salichos, M~K Simon, and O~Keeble.
\newblock Fts3: New data movement service for wlcg.
\newblock {\em J. Phys.: Conf. Ser}, 513(3):032081, 2014.

\bibitem{grafana-FTS}
CERN~{MONIT} team.
\newblock {CERN} {MONIT} {Grafana} {FTS} {Dashboard}.
 
\bibitem{grafana-WLCG-Transfers}
CERN~{MONIT} team.
\newblock {CERN} {MONIT} {Grafana} {WLCG} {Transfers} {Dashboard}.


\bibitem{xrootd-ipv6}
SLAC~{XRootD} team.
\newblock {XRootD System Monitoring Reference}.

%section 4 references
\bibitem{rfc} All Internet Engineering Task Force Requests For Comments (RFC) documents are available
from URLs such as http://www.ietf.org/rfc/rfcNNNN.txt where NNNN is the RFC number, for example {\tt http://www.ietf.org/rfc/rfc2460.txt}

\end{thebibliography}

\end{document}

% end of file template.tex

<div id='footer'><table width='100%'><tr><td class='right'><a href='http://fusioninventory.org/'><span class='copyright'>FusionInventory 9.1+1.0 | copyleft <img src='/glpi/plugins/fusioninventory/pics/copyleft.png'/>  2010-2016 by FusionInventory Team</span></a></td></tr></table></div>
