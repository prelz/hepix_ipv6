\documentclass{webofc}
\usepackage[varg]{txfonts}   % Web of Conferences font
%
% Put here some packages required or/and some personnal commands
%
\usepackage{hyperref}
\hypersetup{colorlinks=false}

%
\begin{document}
%
\title{IPv6-only networking on WLCG}
%
% subtitle is optionnal
%
%%%\subtitle{Do you have a subtitle?\\ If so, write it here}

\author{
  \firstname{Marian} \lastname{Babik}\inst{1}\and
  \firstname{Martin} \lastname{Bly}\inst{2}\and
  \firstname{Nick} \lastname{Buraglio}\inst{3}\and
  \firstname{Tim} \lastname{Chown}\inst{4}\and
  \firstname{Dimitrios} \lastname{Christidis}\inst{1}\and
  \firstname{Ji\v{r}i} \lastname{Chudoba}\inst{5}\and
  \firstname{Phil} \lastname{DeMar}\inst{6}\and
  \firstname{Jos\'e} \lastname{Flix~Molina}\inst{7}\and
  \firstname{Costin} \lastname{Grigoras}\inst{1}\and
  \firstname{Bruno} \lastname{Hoeft}\inst{9}\and
  \firstname{Hiro} \lastname{Ito}\inst{10}\and
  \firstname{David} \lastname{Kelsey}\inst{2}\thanks{\email{david.kelsey@stfc.ac.uk}} \and
  \firstname{Edoardo} \lastname{Martelli}\inst{1}\and
  \firstname{Shawn} \lastname{McKee}\inst{11}\and
  \firstname{Maria del Carmen} \lastname{Misa~Moreira}\inst{1}\and
  \firstname{Raja} \lastname{Nandakumar}\inst{2}\and
  \firstname{Kars} \lastname{Ohrenberg}\inst{7}\and
  \firstname{Francesco} \lastname{Prelz}\inst{12}\and
  \firstname{Duncan} \lastname{Rand}\inst{13}\and
  \firstname{Andrea} \lastname{Sciab\`a}\inst{1}\and
  \firstname{Tim} \lastname{Skirvin}\inst{6}
}

\institute{ 
  European Organization for Nuclear Research (CERN), CH-1211 Geneva 23, Switzerland
\and
  UKRI STFC Rutherford Appleton Laboratory (RAL), Harwell Campus, Didcot OX11 0QX, United Kingdom
\and
  Energy Sciences Network (ESNET), Lawrence Berkeley National Laboratory, 1 Cyclotron Rd, Berkeley CA 94720, United States of America
\and
  JISC, Lumen House, Library Avenue, Harwell Campus, Didcot OX11 0SG, United Kingdom
\and
  Institute of Physics, Academy of Sciences of the Czech Republic, Na Slovance 2 182 21 Prague 8, Czech Republic
\and
  Fermi National Accelerator Laboratory (FNAL), P.O. Box 500, Batavia IL 60510, United States of America
\and
  Centro de Investigaciones Energ\'eticas, Medioambientales y Tecnol\'ogicas (CIEMAT), Av. Complutense 40, E-28040 Madrid, Spain
\and
  Karlsruhe Institute of Technology (KIT), Hermann-von-Helmholtz-Platz 1, D-76344 Eggenstein-Leopoldshafen, Germany 
\and
  Brookhaven National Laboratory (BNL), 98 Rochester St., Upton NY 11973, United States of America
\and
  University of Michigan, 500 S State S, Ann Arbor MI 48109, United States of America
\and
  Deutsches Elektronen-Synchrotron (DESY), Notkestra\ss e 85, D-22607 Hamburg, Germany
\and
  INFN, Sezione di Milano, via G. Celoria 16, I-20133 Milano, Italy
\and
  Imperial College London, South Kensington Campus, London SW7 2AZ, United Kingdom
}

\abstract
{The transition of WLCG storage services to dual-stack IPv4/IPv6 is nearing completion after more
than 5 years, thus enabling the use of IPv6-only CPU resources as agreed by the WLCG Management
Board and presented by us at earlier CHEP conferences. Much of the data is transferred by the LHC
experiments over IPv6. All Tier-1 storage and over 90% of Tier-2 storage is now IPv6-enabled, yet we
still see IPv4 transfers happening when both endpoints have IPv6 available or when remote data is
accessed over the network from worker nodes.
The monitoring and tracking of all data transfers is essential, together with the ability to understand
the relative use of IPv6 and IPv4. This paper presents the status of monitoring IPv6 data flows within
WLCG and plans to improve the ability to distinguish between IPv6 and IPv4. Furthermore, the
Research Networking Technical Working Group has identified marking the IPv6 packet header as one
approach for understanding complex large data flows. This provides another driver for a full
transition to the use of IPv6 in WLCG data transfers.
The agreed endpoint of the WLCG transition to IPv6 remains the deployment of IPv6-only services,
thereby removing the complexity and security concerns of operating dual stacks. The working group
is identifying where IPv4 can be removed and investigating the obstacles to the use of IPv6 in WLCG.
Why do transfers between two dual-stack endpoints still use IPv4? This work is presented together
with the obstacles defeated, those remaining, and those outside of our control.}
\maketitle
\section{Introduction}
\label{sec-intro}
%section 1

The HEPiX IPv6 Working Group \cite{ipv6wg} has been investigating the many issues involved in the deployment and use of
IPv6 in HEP in general and more specifically in WLCG. The group's paper at CHEP2016 \cite{ipv6chep2016}
presented the status then of the work to allow sites to deploy IPv6-only CPU resources. Driven by the
requirements of the LHC experiments, the WLCG Management Board, in September 2016, had approved our plan
that all WLCG Tier-2 storage services should aim to support dual-stack IPv6/IPv4 by the end of 2018. Since then the
group has worked with others to encourage, support and monitor that transition and to identify and help
solve any technical issues as they arise.

%This paper is organised as follows.  Section 2 presents the current status of the transition for the Tier0/Tier1s, the Tier2s and the Experiment Services.
%Section 3 presents an update on service availability and network monitoring and also reports on the fraction of FTS data transfers currently taking place over IPv6.
%Finally section 4 contains future plans and conclusions. 

 

% Etc. etc. etc.
%For one-column wide figures use syntax of figure~\ref{fig-1}
%\begin{figure}[h]
%% Use the relevant command for your figure-insertion program
%% to insert the figure file.
%\centering
%\includegraphics[width=1cm,clip]{tiger}
%\caption{Please write your figure caption here}
%\label{fig-1}       % Give a unique label
%\end{figure}
%
%For two-column wide figures use syntax of figure~\ref{fig-2}
%\begin{figure*}
%\centering
%% Use the relevant command for your figure-insertion program
%% to insert the figure file. See example above.
%% If not, use
%\vspace*{5cm}       % Give the correct figure height in cm
%\caption{Please write your figure caption here}
%\label{fig-2}       % Give a unique label
%\end{figure*}
%
%For figure with sidecaption legend use syntax of figure
%\begin{figure}
%% Use the relevant command for your figure-insertion program
%% to insert the figure file.
%\centering
%\sidecaption
%\includegraphics[width=5cm,clip]{tiger}
%\caption{Please write your figure caption here}
%\label{fig-3}       % Give a unique label
%\end{figure}
%
%For tables use syntax in table~\ref{tab-1}.
%\begin{table}
%\centering
%\caption{Please write your table caption here}
%\label{tab-1}       % Give a unique label
%% For LaTeX tables you can use
%\begin{tabular}{lll}
%\hline
%first & second & third  \\\hline
%number & number & number \\
%number & number & number \\\hline
%\end{tabular}
%% Or use
%\vspace*{5cm}  % with the correct table height
%\end{table}
%
% BibTeX or Biber users please use (the style is already called in the class, ensure that the "woc.bst" style is in your local directory)
% \bibliography{name or your bibliography database}
%
% Non-BibTeX users please use
%
%\begin{thebibliography}{}
%
% and use \bibitem to create references.
%
%\bibitem{rfc} All Internet Engineering Task Force Requests For Comments (RFC) do
%cuments are available
%from URLs such as http://www.ietf.org/rfc/rfcNNNN.txt where NNNN is the RFC numb
%er, for example {\tt http://www.ietf.org/rfc/rfc2460.txt}
% Format for Journal Reference
%Journal Author, Journal \textbf{Volume}, page numbers (year)
% Format for books
%\bibitem{RefB}
%Book Author, \textit{Book title} (Publisher, place, year) page numbers
% etc
%\end{thebibliography}
%
\begin{thebibliography}{}
%
% and use \bibitem to create references.
%
% Format for Journal Reference
%Journal Author, Journal \textbf{Volume}, page numbers (year)
% Format for books
%\bibitem{RefB}
%Book Author, \textit{Book title} (Publisher, place, year) page numbers
% etc


%section 1 references
%\bibitem{ipv6wg} The HEPiX IPv6 Working Group, http://hepix-ipv6.web.cern.ch
\bibitem{ipv6wg}
S. Campana et al, J. Phys. Conf. Ser. {\bf513}, 062026 (2014)



%\bibitem{ipv6chep2016} 
% M. Babik et al, J. Phys. Conf. Ser. {\bf898}, 082033 (2017)
\bibitem{ipv6chep2018} 
M. Babik et al, J. Phys. Conf. Ser. {\bf214}, 08010 (2019)

\bibitem{eos} A. J. Peters et al, J. Phys. Conf. Ser. {\bf664}, 042042 (2015)


%section 2 references

\bibitem{rfc} 
All Internet Engineering Task Force Requests For Comments (RFC) documents are available
from URLs such as https://www.ietf.org/rfc/rfcNNNN.txt where NNNN is the RFC number, for example {\tt https://www.ietf.org/rfc/rfc2460.txt}



%section 2 intro para

\bibitem{ipv6chep2015} 
J. Bernier et al, J. Phys. Conf. Ser. {\bf664}, 052018 (2015)

%section 2.1 Tier 0/1
\bibitem{fts3}
A.~A. Ayllon et al, 
%M~Salichos, M~K Simon, and O~Keeble.
%\newblock Fts3: New data movement service for wlcg.
J. Phys. Conf. Ser. {\bf 513}, 032081 (2014)

%section 2.2 Tier 2

%section 2.3 LHCOPN
\bibitem{opnone}
E. Martelli et al, J. Phys. Conf. Ser. {\bf 664}, 052025 (2015)

%section 2.4 Data transfers



%section 2 Experiments
%\bibitem{alien} 
%S. Bagnasco et al, J. Phys. Conf. Ser. {\bf119(6)}, 062012 (2008)

%\bibitem{xrootd}
%L. Bauerdick et al, J. Phys. Conf. Ser. {\bf 396} 042009 (2012)

%\bibitem{jalien}
%A. Grigora et al, J. Phys. Conf. Ser. {\bf523}, 012010 (2014)

%\bibitem{glideinwms} 
%http://iopscience.iop.org/article/10.1088/1742-6596/119/6/062044
%I. Sfiligoi, J. Phys. Conf. Ser. {\bf119(6)}, 062044 (2008)

%\bibitem{htcondor}
%D. Thain et al, Concurrency - Practice and Experience, {\bf 7(2-4)}, 323 (2005)

%\bibitem{dirac} A. Tsaregorodtsev et al, J. Phys. Conf. Ser. {\bf513}, 032096 (2014)

%LHCb : LHCb collaboration, A. A. Alves Jr. et al., The LHCb detector at the LHC, JINST 3 (2008) S08005

%section 3 references

%section 3.1
\bibitem{ipv6trans}
M. Nikkhah and R. Gu\'erin, 
IEEE/ACM Transactions on Networking, {\bf 24(4)}, 2291 (2016)

\bibitem{jool}
https://www.jool.mx

%section 3.2

\bibitem{RefLHCOPNEv4v6}
LHCOPN and LHCONE traffic flows on the CERN border routers, 
https://twiki.cern.ch/twiki/bin/view/LHCOPN/LHCOPNEv4v6Traffic


%section 3.3
\bibitem{frontier}
http://frontier.cern.ch/

%\bibitem{sam}
%A. Aimar et al, J. Phys. Conf. Ser. {\bf 898(9)}, 092033 (2017)
%A~Aguado Corman, P~Andrade, S~Belov, J~Delgado Fernandez, B~Garrido Bear, M~Georgiou, E~Karavakis, L~Magnoni, R~Rama Ballesteros, H~Riahi, J~Rodriguez Martinez, P~Saiz, and D~Zolnai.
%\newblock Unified monitoring architecture for it and grid services.


%\bibitem{etf}
%Marian Babik, CERN,
%\newblock Experiments {Test} {Framework} ({ETF}).
%http://etf.cern.ch/docs/latest/

%\bibitem{perfsonar} 
%A. Hanemann et al, 
%Jeff~W. Boote, Eric~L. Boyd, J{\'e}r{\^o}me Durand, Loukik
 %Kudarimoti, Roman {\L}apacz, D.~Martin Swany, Szymon Trocha, and Jason Zurawski.
%\newblock Perfsonar: A service oriented architecture for multi-domain network monitoring.
%\newblock In Boualem Benatallah, Fabio Casati, and Paolo Traverso, editors,
  %{\em Service-Oriented Computing - ICSOC 2005}, pages 241--254, Berlin, Heidelberg, 2005. Springer Berlin Heidelberg.

%\bibitem{wlcg-NTWG}
%S.~McKee et al, J. Phys. Conf. Ser. {\bf 664(5)} 052003 (2015)
%S.~Campana, A.~Di Girolamo, T.~Wildish, J.~Closier, S.~Roiser, C.~Grigoras, I.~Vukotic, M.~Salichos, Kaushik De, V.~Garonne, J.A.D. Cruz, A.~Forti, C.J. Walker, D.~Rand, A.~de~Salvo, E.~Mazzoni, I.~Gable, F.~Chollet, L.~Caillat, F.~Schaer, Hsin-Yen Chen, U.~Tigerstedt, G.~Duckeck, B.~Hoeft, A.~Petzold, F.~Lopez, J.~Flix, S.~Stancu, J.~Shade, M.~O'Connor, V.~Kotlyar, and J.~Zurawski.
%\newblock Integrating network and transfer metrics to optimize transfer efficiency and experiment workflows.


%\bibitem{psmad}  
%perfSONAR Consortium,
%\newblock {perfSONAR} {Monitoring and Debugging Dashboard (MADDASH)}.
%http://psmad.grid.iu.edu/toolkit/
%http://psmad.grid.iu.edu/maddash-webui/
%index.cgi?dashboard=OPN%20Mesh%20Config

%\bibitem{grafana-ipv6}
%CERN monitoring team,
%\newblock {CERN} {MONIT} {Grafana} {Dashboard}.
%https://monit-grafana.cern.ch/





%\bibitem{RefLHCOPNEv4v6}
%LHCOPN and LHCONE traffic flows on the CERN border routers, 
%https://twiki.cern.ch/twiki/bin/view/LHCOPN/LHCOPNEv4v6Traffic


%\bibitem{grafana-FTS}
%CERN~{MONIT} team.
%\newblock {CERN} {MONIT} {Grafana} {FTS} {Dashboard}.
 
%\bibitem{grafana-WLCG-Transfers}
%CERN~{MONIT} team.
%\newblock {CERN} {MONIT} {Grafana} {WLCG} {Transfers} {Dashboard}.


%\bibitem{xrootd-ipv6}
%SLAC~{XRootD} team.
%\newblock {XRootD System Monitoring Reference}.


%\bibitem{sam} A. Aimar et al, J. Phys. Conf. Ser. {\bf 898}, 092033 (2017)
%A.A. Corman, P. Andrade, S. Belov, J.D. Fernandez, B.G. Bear, M. Georgiou,E. Karavakis, L. Magnoni, R.R. Ballesteros et al., Unified Monitoring Architecture for IT and Grid Services


%\bibitem{etf} M. Babik,Experiments Test Framework (ETF), https://etf.cern.ch/docs

%\bibitem{perfsonar} A.  Hanemann et al,  
%J.W.  Boote,  E.L.  Boyd,  J.  Durand,  L.  Kudarimoti,  R.  Łapacz,  D.M.Swany, S. Trocha, J. Zurawski, PerfSONAR: A Service Oriented Architecture for Multi-domain Network Monitoring, inService-Oriented Computing  ICSOC 2005, edited by B. Benatallah et al (Springer Berlin Heidelberg, Berlin, Heidelberg,2005), pp. 241-254, ISBN 978-3-540-32294-8

%\bibitem{wlcg-NTWG}  S. McKee, M. Babik, S. Campana, A.D. Girolamo, T. Wildish, J. Closier, S. Roiser,C. Grigoras, I. Vukotic, M. Salichos et al., Integrating network and transfer metrics to optimize transfer efficiency and experiment workflows, Journal of Physics:  Conference Series664,052003 (2015)

%\bibitem{psmad} perfSONAR Consortium,perfSONAR Monitoring and Debugging Dashboard (MAD-DASH),http://psmad.opensciencegrid.org/maddash-webui/index.cgi

%\bibitem{grafana-ipv6} CERN  MONIT  Grafana  Dashboard,http://monit-grafana-open.cern.ch/d/000000809/perfsonar-ipv6



%section 4 references


\end{thebibliography}


%
\end{document}
%
% end of file template.tex

