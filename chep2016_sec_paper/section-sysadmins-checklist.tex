
This section contains the IPv6 security checklist for sysadmins and site networking/security teams. This presents the key IPv6 security issues to be addressed by WLCG sites as a starting point. More information is available in the documents referenced here or in the Introduction. 



\begin{enumerate}


\item {\bf Make an addressing plan}\\
One of the most important design decisions for a site networking team creating an IPv6 deployment plan is to create a well thought out management plan for their IPv6 address space. This needs to include consideration as to how to manage a dual-stack network. IPv6 address space (typically a /48 - default size as in RFC3177) will have been allocated to the site by its NREN or other ISP. Consideration needs to be given to the routing and switching design of the network. The number of subnets, the routing architecture, and the address allocation within subnets etc. all need to be included. 

See http://www.internetsociety.org/deploy360/resources/ipv6-address-planning-guidelines-for-ipv6-address-allocation/   and https://www.ripe.net/support/training/material/IPv6-for-LIRs-Training-Course/Preparing-an-IPv6-Addressing-Plan.pdf  


\item {\bf Decide whether to use DHCPv6 or SLAAC+DDNS} \\
The second most important decision (and very much linked to the one above) to be made by a site networking team is whether or not to use one of the important new features of IPv6, i.e. the end-system use of  IPv6 Stateless Address Autoconfiguration (see RFC4862).  And use of dynamic DNS.  You are likely to prefer that server systems have fixed addresses (either manual or DHCPv6).



\item {\bf Ensure all security/network monitoring/logging are IPv6-capable}\\
Important for networking teams, security teams and also end systems tools for sysadmins. All monitoring and logging tools (commercial, open-source, home written) need to be evaluated and tested for operation on IPv6. New longer addresses and multiple addresses per network card. Do they work in a dual-stack environment and can they simultaneously monitor both stacks.  What about tools analysing log files - does parsing work?


\item {\bf Filter IPv6 packets that enter and leave your network/system}\\
IPv4-only networks will have end systems where IPv6 is enabled by default. May cause big security problems with IDS and Firewalls not handling IPv6 traffic correctly. If you don't want IPv6 best to turn it off and/or filter it at both the network and system level.



For filtering of ICMPv6 packets - see next topic.


\item {\bf Filter ICMPv6 messages wisely} \\
Many ICMPv6 messages have an essential role in establishing or maintaining IPv6 communication. Some ICMPv6 messages can therefore not be blocked (unlike in IPv4). Other types of ICMPv6 if allowed through the site firewall may lead to security issues. See RFC 4890 for a full discussion and for advice.

\item {\bf Allow special-purpose headers only if needed} \\
Extension Headers open up your end system and site to all sorts of vulnerabilities. RFC

Filter packets with inappropriate Extension Headers. Extension headers complicate the task of firewall ACLs. The site needs to verify whether its ACLs also process extension headers (to parse them to find the upper-layer payloads) and to block unwanted extension headers see RFC5095 and RFC7045.


\item {\bf Use synchronised IPv4/v6 access rules} \\
For dual-stack networks MUCH much better to have identical firewall rules (site and end-system) for both stacks. Making them different can cause problems with



\item {\bf Deploy RA-Guard or otherwise deal with Rogue RAs} \\
Neigbor Discovery and Router discovery are two important new features of IPv6. Opens up to several different security problems. Rogue routers can send out false router announcements (RAs) to persuade end systems to send packets to them for routing allowing for lack of privacy and man in the middle attacks.  Several ways of addressing this, one of which is to deploy RA-Guard.


\item {\bf Do not be tempted by transition technologies} \\
By this  we mean think very carefully before deciding to use or allow the use of tunnelling technologies.  Dual-stack systems are the simplest approach. NAT64 is being used by some WLCG sites but we do not in general recommend this unless ...
problems with tunnels and protocol translations are ...

\item {\bf Filter/disable IPv6-on-IPv4 tunnels} \\
We recommend not using such tunnels (see above).  So we suggest that sites should filter or disable these.


\end{enumerate}



