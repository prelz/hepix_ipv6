\documentclass[a4paper]{jpconf}
\usepackage{graphicx}
\usepackage{color}
\usepackage{array}
\usepackage{enumerate}
\usepackage{framed}

\begin{document}
\title{IPv6 Security}

\author{M~Babik$^1$, J~Chudoba$^2$, A~Dewhurst$^3$, T~Finnern$^4$, T~Froy$^5$,
        C~Grigoras$^1$, K~Hafeez$^3$, B~Hoeft$^6$, T~Idiculla$^3$, D~P~Kelsey$^3$,  
        F~L\'opez~Mu\~noz$^{7,8}$, E~Martelli$^1$, R~Nandakumar$^3$, 
        K~Ohrenberg$^4$, F~Prelz$^{9}$, D~Rand$^{10}$, 
        A~Sciab\`a$^1$, D~Traynor$^5$, U~Tigerstedt$^{11}$ and R~Wartel$^1$}

%\address{$^1$ IN2P3 Computing Centre, Boulevard du 11 Novembre 1918, F-69622 Villeurbanne Cedex, France}
\address{$^1$ CERN, CH-1211 Gen\`eve 23, Switzerland}
\address{$^2$ Institute of Physics, Academy of Sciences of the Czech Republic Na Slovance 2 182 21 Prague 8, Czech Republic}
\address{$^3$ STFC Rutherford Appleton Laboratory, Harwell Campus, Didcot, Oxfordshire OX11 0QX, United Kingdom}
\address{$^4$ Deutsches Elektronen-Synchrotron DESY, Notkestra\ss e 85, D-22607 Hamburg, Germany}
\address{$^5$ Queen Mary University of London, Mile End Road, London E1 4NS, United Kingdom}
\address{$^6$ Karlsruher Institut f\"ur Technologie, Hermann-von-Helmholtz-Platz 1, D-76344 Eggenstein-Leopoldshafen, Germany}
\address{$^7$ Port d'Informaci\'o Cient\'ifica, Campus UAB, Edifici D, E-08193 Bellaterra, Spain}
\address{$^8$ Also Centro de Investigaciones Energ\'eticas, Medioambientales y Tecnol\'ogicas (CIEMAT), Madrid, Spain}
\address{$^9$ INFN, Sezione di Milano, via G. Celoria 16, I-20133 Milano, Italy}
\address{$^{10}$ Imperial College London, South Kensington Campus, London SW7 2AZ, United Kingdom}
\address{$^{11}$ CSC Tieteen Tietotekniikan Keskus Oy, P.O. Box 405, FI-02101 Espoo}
%\address{$^3$ Fermi National Accelerator Laboratory, Batavia, Il 60510, U.S.A.}
%\address{$^3$ Institute of High Energy Physics, 19B Yuquanlu, Shijingshan District, 100049 Beijing, China} 
%\address{$^{9}$ The University of Oxford, Denys Wilkinson Building, Keble Road, Oxford OX1 3RH, United Kingdom}
%\address{$^{12}$ California Institute of Technology, Pasadena, Ca 91125, U.S.A.}



\ead{david.kelsey@stfc.ac.uk, ipv6@hepix.org}

\begin{abstract}
IPv4 network addresses are running out and the deployment of IPv6 networking in many places is now well underway. Following the work of the HEPiX IPv6 Working Group, a growing number of sites in the Worldwide Large Hadron Collider Computing Grid (WLCG) have deployed dual-stack IPv6/IPv4 services. The aim of this is to support the use of IPv6-only clients, i.e. worker nodes, virtual machines or containers.
\par
The IPv6 networking protocols while they do contain features aimed at improving security also bring new challenges for operational IT security. We have spent many decades understanding and fixing security problems and concerns in the IPv4 world. Many WLCG IT support teams have only just started to consider IPv6 security and they are far from ready to follow best practice, the guidance for which is not easy to distil. The lack of maturity of IPv6 implementations together with the increased complexity of some of the protocol standards while noting that the new protocol stack allows for many of the same attack vectors as IPv4, raise many new issues for operational security teams.
\par
The HEPiX IPv6 Working Group is producing guidance on best practices in this area. This paper considers some of the security concerns for WLCG in an IPv6 world and presents the HEPiX IPv6 working group guidance for the system administrators who manage IT services  on the WLCG distributed infrastructure, for their related site security and networking teams and for developers and software engineers working on WLCG applications.
\end{abstract}

\section{Introduction}
The world's Regional Internet Registries are rapidly running out of available IPv4 addresses and the 
general slow transition to IPv6 continues. The Worldwide Large Hadron Collider Grid (WLCG) and the LHC experiments 
may soon have access to worker nodes or virtual machines possessing only an IPv6-routable address. The HEPiX
IPv6 Working Group [Ref 1] has been investigating the many issues feeding into the move to the use of IPv6 in HEP and WLCG.
The group's paper at CHEP2013 [Ref 2] described the aims of the group and the testing of dual-stack IPv6/IPv4 
services that had been completed at that point. In the last 18 months the group has worked more closely with the
4 major LHC experiments and identified the main use case for the support of IPv6-only clients on WLCG. The groups
activities, including testing of dual-stack data storage services, during the last 18 months are presented in this 
paper together with its future plans.

\section{Some IPv6 security issues}
IPv6 Security issues

New features

Many more ICMP message types! Cannot filter all of them (MTU discovery has to work). Must filter some of them. RFC4890 gives advice

New methods for autoconfiguring addresses, routes, DNS. Good for the end-user. Must do something against rogue Router Advertisements (see RFC6104)

Longer IP addresses. Hey, everyone knows that. They may slow down brute force scans. But no bad guy is that crude...

Cannot fragment packets en-route.  Minimum MTU: 1280. But you can still hurt yourself and send small fragments if you wish. Some good news, at least

Not really a feature of IPv6 proper, but much of the network stack and application code is enticingly fresh!

Transitional technologies (e.g. tunnels) have intrinsic vulnerabilities but don't need to be there forever...


Business as usual

As long as all network monitoring and administration tools are up-to-date and (therefore) aware of IPv6.

Broadcasts and Multicasts are still there, with a vengeance.
Can still use IP headers for out-of-band communications.
Can still pollute Ethernet address discovery (ND instead of ARP).
Can still run a rogue DHCP server.
Can still try forging and injecting packets into the local network. 
Upper-layer protocols did not change!

\section{Checklist for WLCG site system administrators and networking teams}
(This Francesco's text - modify this!)
When applications developed in the golden
era of IPv4-only Internet face the transition to IPv6 the brunt of the
work often falls on the shoulders of developers, who often belong to a different
generation as the original authors. Figure \ref{fig:pseudocode} tries to
visualise the {\it extent} of the changes that the core code of any
IP-capable application undergoes in the transition. In addition to
the extensively different syntax, there is a
fundamental $1\rightarrow N$ change here:
no IP endpoint can be satisfied with handling just {\it one}
IP address (as any public IPv6 endpoint communicates via the
public and on the link-local address at least), but loops and address
ordering start appearing everywhere. We identify the following implications of this
fundamental fact on developers' practice, in rough descending order
of importance:
\begin{enumerate}
\item The {\it syntactic} change of the core IP networking code in the
      IPv4$\rightarrow$IPv6 transition is large enough to oftentime
      justify the refactoring of larger portions of code. The 
      {\it semantic} $1\rightarrow N$ change may be {\it forcing}
      some rethiking at the design level. A possible temptation here
      is to provide parallel sections of code that handle the IPv6 case
      only: a few other reasons why this may not be a good idea are
      listed below. In any case, there is an implicit expectation that 
      a change that should be affecting the {\it transport} layer of
      the network only should cause no ripple in the upper layers, i.e.
      that the perceived responsiveness, performance and reliability
      of the code remain unchanged. {\it Extensive} stress-testing should
      therefore be planned on IPv6-ported code. 
\item Code that binds and connects IP sockets is suddenly faced with 
      making choices that used to be delegated to the operating system or
      networking-capable libraries. Lists of addresses may be received in
      a given order, but it's now the responsibility of the socket-handling
      code to iterate and re-iterate on the list, handle exceptions and
      possibly operate in parallel on various entries
      to implement some form of 'happy-eyeballs'\footnote{See RFC6555 \cite{rfc}.}
      alghorithm. As the ordering of both source and destination addresses
      established at the system level by the system
      administrator\footnote{Via {\tt /etc/gai.conf}, {\tt ip addrlabel} or their equivalent.}
      may have security implications, developers should go the extra mile to 
      keep that ordering even if they have to reshuffle the list for any reason.
\item Fresh new code that hasn't been tested broadly and in the 
      wild is {\it per se} attractive to anyone looking for
      malicious exploits. Especially in the case where IPv6-specific
      code or processes are developed for {\it parallel} deployment with
      well-proven IPv4 code, one should make sure that any security measure,
      filter or wisdom that was included in the code for the IPv4 case isn't
      simply forgotten for IPv6. While it may not be immediately apparent,
      {\it all} constructs that are meaningful for IPv4 have their
      translation or counterpart for IPv6.
\end{enumerate}


\section{Checklist for developers}
\begin{figure}
\begin{framed}
{\tt\small
struct hostent  *resolved\_name=NULL;\\
struct servent  *resolved\_serv=NULL;\\
struct protoent *resolved\_proto=NULL;\\
static char     *dest\_host="some.ip.host", *dest\_serv="ipservice";\\
struct sockaddr\_in destination;\\ 
resolved\_host = gethostbyname(dest\_host);\\
resolved\_serv = getservbyname(dest\_serv, NULL);\\
if (resolved\_host != NULL \&\& resolved\_serv != NULL) \{\\
\qquad destination.sin\_family = resolved\_name->h\_addrtype;\\
\qquad destination.sin\_port = htons(resolved\_serv->s\_port);\\
\qquad memcpy(\&destination.sin\_addr, resolved\_host->h\_addr\_list[0],\\
\qquad\qquad resolved\_host->h\_length);\\
\qquad resolved\_proto = getprotobyname(resolved\_serv->s\_proto)\\
\qquad if (resolved\_proto != NULL) \{\\
\qquad\qquad int fd = socket(AF\_INET, SOCK\_STREAM, resolved\_proto->p\_proto);\\
\qquad\qquad connect(fd, \&destination, sizeof(destination));\\
\qquad\qquad /* Check for errors, connect, etc... */\\
\qquad \}\\
\}
}
\end{framed}
\par
\begin{framed}
{\tt\small
struct addrinfo ai\_req, *ai\_ans, *cur\_ans;\\
static char     *dest\_host="some.ip.host", *dest\_serv="ipservice";\\
ai\_req.ai\_flags = 0;\\
ai\_req.ai\_family = PF\_UNSPEC;\\
ai\_req.ai\_socktype = SOCK\_STREAM;\\
ai\_req.ai\_protocol = 0; /* Any protocol is OK */\\
if (getaddrinfo(dest\_host, dest\_serv, \&ai\_req, \&ai\_ans) != 0) \{\\
\qquad for (cur\_ans = ai\_ans; cur\_ans != NULL; cur\_ans = cur\_ans->ai\_next) \{\\
\qquad\qquad int fd = socket(cur\_ans->ai\_family, cur\_ans->ai\_socktype,\\
\qquad\qquad\qquad cur\_ans->ai\_protocol);\\
\qquad\qquad connect(fd\_socket,cur\_ans->ai\_addr,cur\_ans->ai\_addrlen);\\
\qquad\qquad /* Check for errors - This loop has the ability to change the */\\
\qquad\qquad /* order of the getaddrinfo results! */\\
\qquad \}\\
\}
}
\end{framed}
\caption{C code snippets showing how the basic IP service resolution and connection 
 changes from legacy IPv4-only to a dual-stack or IPv6-only environment. This represents
 the zeroth-order porting effort for much IPv4-only code. The newer structure
 is more terse, but the changes are extensive enough, both syntactically and
 semantically, to probably trigger the refactoring of much larger sections of code.}
\label{fig:pseudocode}
\end{figure}
When applications developed in the golden
era of IPv4-only Internet face the transition to IPv6 the brunt of the
work often falls on the shoulders of developers, who often belong to a different
generation as the original authors. Figure \ref{fig:pseudocode} tries to
visualise the {\it extent} of the changes that the core code of any
IP-capable application undergoes in the transition. In addition to
the extensively different syntax, there is a
fundamental $1\rightarrow N$ change here:
no IP endpoint can be satisfied with handling just {\it one}
IP address (as any public IPv6 endpoint communicates via the
public and on the link-local address at least), but loops and address
ordering start appearing everywhere. We identify the following implications of this
fundamental fact on developers' practice, in rough descending order
of importance:
\begin{enumerate}
\item The {\it syntactic} change of the core IP networking code in the
      IPv4$\rightarrow$IPv6 transition is large enough to oftentime
      justify the refactoring of larger portions of code. The 
      {\it semantic} $1\rightarrow N$ change may be {\it forcing}
      some rethiking at the design level. A possible temptation here
      is to provide parallel sections of code that handle the IPv6 case
      only: a few other reasons why this may not be a good idea are
      listed below. In any case, there is an implicit expectation that 
      a change that should be affecting the {\it transport} layer of
      the network only should cause no ripple in the upper layers, i.e.
      that the perceived responsiveness, performance and reliability
      of the code remain unchanged. {\it Extensive} stress-testing should
      therefore be planned on IPv6-ported code. 
\item Code that binds and connects IP sockets is suddenly faced with 
      making choices that used to be delegated to the operating system or
      networking-capable libraries. Lists of addresses may be received in
      a given order, but it's now the responsibility of the socket-handling
      code to iterate and re-iterate on the list, handle exceptions and
      possibly operate in parallel on various entries
      to implement some form of 'happy-eyeballs'\footnote{See RFC6555 \cite{rfc}.}
      alghorithm. As the ordering of both source and destination addresses
      established at the system level by the system
      administrator\footnote{Via {\tt /etc/gai.conf}, {\tt ip addrlabel} or their equivalent.}
      may have security implications, developers should go the extra mile to 
      keep that ordering even if they have to reshuffle the list for any reason.
\item Fresh new code that hasn't been tested broadly and in the 
      wild is {\it per se} attractive to anyone looking for
      malicious exploits. Especially in the case where IPv6-specific
      code or processes are developed for {\it parallel} deployment with
      well-proven IPv4 code, one should make sure that any security measure,
      filter or wisdom that was included in the code for the IPv4 case isn't
      simply forgotten for IPv6. While it may not be immediately apparent,
      {\it all} constructs that are meaningful for IPv4 have their
      translation or counterpart for IPv6.
\end{enumerate}


\section{Summary}
% Summary

In this paper, we have discussed some of the security concerns that arise from the deployment of the IPv6 networking protocols. In particular, we have presented our IPv6 security checklist for sysadmins and site networking/security teams at WLCG sites. We have also presented a checklist for WLCG/HEP application developers and software engineers. We welcome feedback from sites and
developers on the contents of these lists according to their experiences during the transition.  HEP IT support staff have spent many decades understanding and fixing security problems and concerns in the IPv4 world. We confidently predict that securing IPv6 will take an equally long time!

\section{Acknowledgements}
% Acknowledgements

The authors acknowledge the contributions to this work made by former members of the HEPiX IPv6 Working Group and other colleagues within WLCG. In particular we express our thanks to Raul Lopes, Simon Furber and colleagues at Brunel University London for their extensive testing of IPv6-only compute nodes. We also thank Edgar Fajardo Hernandez from the University of California, San Diego for his work on the validation of IPv6 and glideinWMS for the CMS experiment.

The authors also acknowledge the support and collaboration of many other colleagues in their respective institutes, experiments and IT Infrastructures, together with the funding received by these from many different sources. 

These include but are not limited to the following:

The Worldwide LHC Computing Grid (WLCG) project is a global collaboration of more than 170 computing centres in 42 countries, linking up national and international grid infrastructures. Funding is acknowledged from many national funding bodies and we acknowledge the support of several operational infrastructures including EGI, OSG and NDGF/NeIC.

EGI acknowledges the funding and support received from the European Commission and the many National Grid Initiatives and other members. The EGI-Engage project is co-funded by the European Commission (grant number 654142).

Authors from the UK acknowledge the funding and support received from the Science and Technology Facilities Council via the GridPP project.





% Add other sections as appropriate...

\par
\begin{thebibliography}{1}
%
% and use \bibitem to create references.
%
% Format for Journal Reference
%Journal Author, Journal \textbf{Volume}, page numbers (year)
% Format for books
%\bibitem{RefB}
%Book Author, \textit{Book title} (Publisher, place, year) page numbers
% etc


%section 1 references
\bibitem{ipv6wg} The HEPiX IPv6 Working Group web site is to be found at {\tt http://hepix-ipv6.web.cern.ch}

\bibitem{ipv6chep2016} 
Babik M et al 2016 Deployment of IPv6-only CPU resources at WLCG sites {\it J. Phys.: Conf. Ser. {\bf898} 082033}

%section 2 references
%section 2 Tier 0/1

%section 2 Tier 2

%section 2 Experiments
\bibitem{alien}  http://cds.cern.ch/record/1171677?ln=en

\bibitem{jalien} http://cds.cern.ch/record/2026281?ln=en

\bibitem{glideinwms} 
http://iopscience.iop.org/article/10.1088/1742-6596/119/6/062044

\bibitem{htcondor}
Douglas Thain, Todd Tannenbaum, and Miron Livny.
\newblock Distributed computing in practice: the condor experience.
\newblock {\em Concurrency - Practice and Experience}, 17(2-4):323--356, 2005.

\bibitem{dirac} Dirac :  A Tsaregorodtsev and the Dirac Project 2014 J. Phys.: Conf.
Ser. 513 032096
LHCb : LHCb collaboration, A. A. Alves Jr. et al., The LHCb detector at
the LHC, JINST 3 (2008) S08005

%section 3 references
\bibitem{sam}
A~Aimar et al 
%A~Aguado Corman, P~Andrade, S~Belov, J~Delgado Fernandez, B~Garrido Bear, M~Georgiou, E~Karavakis, L~Magnoni, R~Rama Ballesteros, H~Riahi, J~Rodriguez Martinez, P~Saiz, and D~Zolnai.
\newblock Unified monitoring architecture for it and grid services.
\newblock {\em Journal of Physics: Conference Series}, 898(9):092033, 2017.

\bibitem{etf}
Marian Babik.
\newblock Experiments {Test} {Framework} ({ETF}).

\bibitem{perfsonar} 
Andreas Hanemann, Jeff~W. Boote, Eric~L. Boyd, J{\'e}r{\^o}me Durand, Loukik
  Kudarimoti, Roman {\L}apacz, D.~Martin Swany, Szymon Trocha, and Jason
  Zurawski.
\newblock Perfsonar: A service oriented architecture for multi-domain network
  monitoring.
\newblock In Boualem Benatallah, Fabio Casati, and Paolo Traverso, editors,
  {\em Service-Oriented Computing - ICSOC 2005}, pages 241--254, Berlin,
  Heidelberg, 2005. Springer Berlin Heidelberg.

\bibitem{wlcg-NTWG}
S.~McKee, M.~Babik et al 
%S.~Campana, A.~Di Girolamo, T.~Wildish, J.~Closier, S.~Roiser, C.~Grigoras, I.~Vukotic, M.~Salichos, Kaushik De, V.~Garonne, J.A.D. Cruz, A.~Forti, C.J. Walker, D.~Rand, A.~de~Salvo, E.~Mazzoni, I.~Gable, F.~Chollet, L.~Caillat, F.~Schaer, Hsin-Yen Chen, U.~Tigerstedt, G.~Duckeck, B.~Hoeft, A.~Petzold, F.~Lopez, J.~Flix, S.~Stancu, J.~Shade, M.~O'Connor, V.~Kotlyar, and J.~Zurawski.
\newblock Integrating network and transfer metrics to optimize transfer
  efficiency and experiment workflows.
\newblock {\em Journal of Physics: Conference Series}, 664(5):052003, 2015.

\bibitem{psmad}  
perfSONAR Consortium.
\newblock {perfSONAR} {Monitoring and Debugging Dashboard (MADDASH)}.

\bibitem{grafana-ipv6}
CERN~{MONIT} team.
\newblock {CERN} {MONIT} {Grafana} {Dashboard}.

\bibitem{638647551}
A~A Ayllon, M~Salichos, M~K Simon, and O~Keeble.
\newblock Fts3: New data movement service for wlcg.
\newblock {\em J. Phys.: Conf. Ser}, 513(3):032081, 2014.

\bibitem{grafana-FTS}
CERN~{MONIT} team.
\newblock {CERN} {MONIT} {Grafana} {FTS} {Dashboard}.
 
\bibitem{grafana-WLCG-Transfers}
CERN~{MONIT} team.
\newblock {CERN} {MONIT} {Grafana} {WLCG} {Transfers} {Dashboard}.


\bibitem{xrootd-ipv6}
SLAC~{XRootD} team.
\newblock {XRootD System Monitoring Reference}.

%section 4 references
\bibitem{rfc} All Internet Engineering Task Force Requests For Comments (RFC) documents are available
from URLs such as http://www.ietf.org/rfc/rfcNNNN.txt where NNNN is the RFC number, for example {\tt http://www.ietf.org/rfc/rfc2460.txt}

\end{thebibliography}

\end{document}

{
