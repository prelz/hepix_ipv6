\section*{References}

\begin{thebibliography}{1}
\bibitem{ipv6wg} {\tt http://hepix-ipv6.web.cern.ch}
\bibitem{ipv6chep2015} {chep2015 paper}
\bibitem{ipv6chep2016} {chep2016 paper}
\bibitem{CiscoBook}
    Hogg S, Vyncke E - IPv6 Security, Cisco Press 2009, ISBN-13: 978-1-58705-594-2
\bibitem{nist800-119} {NIST SP800-119}
\bibitem{10myths} {\tt http://www.internetsociety.org/deploy360/blog/category/ipv6/security/}
\bibitem{SANSipv6} {\tt https://www.sans.org/reading-room/whitepapers/ \hfill \break detection/complete-guide-ipv6-attack-defense-33904}
\bibitem{ERNWipv6} {\tt https://www.ernw.de/download/ERNW\_Guide\_to\_Securely\_Configure\_Linux\_Servers\_For\_IPv6\_v1\_0.pdf}
\bibitem{rfc} All Internet Engineering Task Force Requests For Comments (RFC) documents are available
from URLs such as http://www.ietf.org/rfc/rfcNNNN.txt where NNNN is the RFC number, for example {\tt http://www.ietf.org/rfc/rfc2460.txt}
\bibitem{planningguides} There is abundant reference material at \\
{\tt http://www.internetsociety.org/deploy360/resources/ipv6-address-planning-guidelines \hfill \break -for-ipv6-address-allocation/} \\
and {\tt https://www.ripe.net/support/training/material/IPv6-for-LIRs-Training-Course/Preparing \hfill \break-an-IPv6-Addressing-Plan.pdf}.
\end{thebibliography}
