Based on the actions initiated at Grid Deployment Board in November and December 2014, to request tier-1s to join the HEPiX-IPv6 working group and to encourage sites moving their production endpoints to dual stack even if this requires concessions of the quotation of their site relatability and site availability. The proposal of the LHC experiment Atlas was to
\begin{itemize}
 \item request that all Tier-1s provide,
	besides an IPv6 peering to LHCOPN,
        a dual stack PerfSONAR machine by first of April 2015
 \item request that Tier-2s provide,
        besides an IPv6 peering to their LHCONE connection,
        a dual stack PerfSONAR machine by August 2015.
\end{itemize}
At the last LHC[OPN/ONE] meeting a proposal was put forward to the effect that
LHCOPN connecting Tier-1 sites to CERN would get IPv6-ready by 1. April 2015 and
LHCONE connecting Tier-[123] sites would become IPv6 ready by August 2015.
No objections to this proposal were presented. The following Tier-1 sites are actively announcing an IPv6 peering to LHCOPN: CH-CERN, DE-KIT, ES-PIC, FR-CCIN2P3, NDGF, NL-T1. IT-INFN-CNAF is currently preparing the IPv6 peering. The group of IPv6 peers over LHCONE is currently even smaller: besides CH-CERN this are the two sites CEA SACLAY and FR-CCIN2P3.
The ipv6 peerings are reflected at the PerfSONAR dualstack dashboard url:\\
{\tt\small http://maddash.aglt2.org/maddash-webui/index.cgi?dashboard=Dual-Stack\%20Mesh\%20Config}\\. It implies that there are still some LHC tier-1 sites, more than two month after the agreed deadline, not offering ipv6 cidr via their LHCONE peering.
