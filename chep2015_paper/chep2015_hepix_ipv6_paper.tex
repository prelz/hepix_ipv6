\documentclass[a4paper]{jpconf}
\usepackage{graphicx}
\usepackage{color}
\usepackage{array}
\usepackage{enumerate}

\begin{document}
\title{The production deployment of IPv6 on WLCG}

\author{J Bernier$^1$, S Campana$^2$, K Chadwick$^3$, J Chudoba$^4$, 
        A Dewhurst$^5$, M Eli\'a\v s$^4$, S Fayer$^6$, T Finnern$^7$,
        C Grigoras$^2$, B Hoeft$^7$, T Idiculla$^5$, D P Kelsey$^5$,  
        F L\'opez Mu\~noz$^9$, E Macmahon $^{10}$, E Martelli$^2$, R Nandakumar$^5$, 
        K Ohrenberg$^7$, F Prelz$^{11}$, D Rand$^6$, 
        A Sciab\`a$^2$, U Tigerstedt$^{12}$, R Voicu$^{13}$, 
        C J Walker$^{14}$ and T Wildish$^{15}$}

\address{$^1$ IN2P3 Computing Centre, Boulevard du 11 Novembre 1918, F-69622 Villeurbanne Cedex, France}
\address{$^2$ CERN, CH-1211 Gen\`eve 23, Switzerland}
\address{$^3$ Fermi National Accelerator Laboratory, Batavia, Il 60510, U.S.A.}
%\address{$^3$ Institute of High Energy Physics, 19B Yuquanlu, Shijingshan District, 100049 Beijing, China} 
\address{$^4$ Institute of Physics, Academy of Sciences of the Czech Republic Na Slovance 2 182 21 Prague 8, Czech Republic}
\address{$^5$ STFC Rutherford Appleton Laboratory, Harwell Oxford, Didcot, Oxfordshire OX11 0QX, United Kingdom}
\address{$^6$ Imperial College London, South Kensington Campus, London SW7 2AZ, United Kingdom}
\address{$^7$ Deutsches Elektronen-Synchrotron, Notkestra\ss e 85, D-22607 Hamburg, Germany}
\address{$^8$ Karlsruher Institut f\"ur Technologie, Hermann-von-Helmholtz-Platz 1, D-76344 Eggenstein-Leopoldshafen, Germany}
\address{$^9$ Port d'Informaci\'o Cient\'ifica, Campus UAB, Edifici D, E-08193 Bellaterra, Spain}
\address{$^{10}$ The University of Oxford, Denys Wilkinson Building, Keble Road, Oxford OX1 3RH, United Kingdom}
\address{$^{11}$ INFN, Sezione di Milano, via G. Celoria 16, I-20133 Milano, Italy}
\address{$^{12}$ CSC Tieteen Tietotekniikan Keskus Oy, P.O. Box 405, FI-02101 Espoo}
\address{$^{13}$ California Institute of Technology, Pasadena, Ca 91125, U.S.A.}
\address{$^{14}$ Queen Mary University of London, Mile End Road, London E1 4NS, United Kingdom}
\address{$^{15}$ Princeton University, Jadwin Hall, Princeton, NJ 08544, U.S.A.}

\ead{david.kelsey@stfc.ac.uk, ipv6@hepix.org}

\begin{abstract}
The world is rapidly running out of IPv4 addresses; the number of IPv6 end systems connected
to the internet is increasing; WLCG and the LHC experiments may soon have access to worker
nodes and/or virtual machines (VMs) possessing only an IPv6 routable address. The HEPiX
IPv6 Working Group ({\tt http://hepix-ipv6.web.cern.ch/}) has been investigating, testing and
planning for dual-stack services on WLCG for several years. Following feedback from our
working group, many of the storage technologies in use on WLCG have recently been made
IPv6-capable. The worldwide HEP computing community now needs to deploy dual-stack
IPv6/IPv4 services on WLCG to allow such use of IPv6-only resources.
This paper will present the IPv6 requirements, tests and plans of each of the four LHC
experiments together with the tests performed both on the IPv6 test-bed and in targeted use
of WLCG production services. This is primarily aimed at IPv6-only worker nodes or VMs
accessing several different implementations of a global dual-stack federated storage service.
The changes required to the operational infrastructure, including monitoring and security, will
be addressed as will the implications for site management. The working group will present
its deployment plan for dual-stack storage services, together with other essential central and
monitoring services, to start during 2015.
\end{abstract}

\section{Introduction}
The world's Regional Internet Registries are rapidly running out of available IPv4 addresses and the 
general slow transition to IPv6 continues. The Worldwide Large Hadron Collider Grid (WLCG) and the LHC experiments 
may soon have access to worker nodes or virtual machines possessing only an IPv6-routable address. The HEPiX
IPv6 Working Group [Ref 1] has been investigating the many issues feeding into the move to the use of IPv6 in HEP and WLCG.
The group's paper at CHEP2013 [Ref 2] described the aims of the group and the testing of dual-stack IPv6/IPv4 
services that had been completed at that point. In the last 18 months the group has worked more closely with the
4 major LHC experiments and identified the main use case for the support of IPv6-only clients on WLCG. The groups
activities, including testing of dual-stack data storage services, during the last 18 months are presented in this 
paper together with its future plans.
\input{section-n-the-testbed.tex}
% Add other sections as appropriate...

\par
\begin{thebibliography}{1}
%
% and use \bibitem to create references.
%
% Format for Journal Reference
%Journal Author, Journal \textbf{Volume}, page numbers (year)
% Format for books
%\bibitem{RefB}
%Book Author, \textit{Book title} (Publisher, place, year) page numbers
% etc


%section 1 references
\bibitem{ipv6wg} The HEPiX IPv6 Working Group web site is to be found at {\tt http://hepix-ipv6.web.cern.ch}

\bibitem{ipv6chep2016} 
Babik M et al 2016 Deployment of IPv6-only CPU resources at WLCG sites {\it J. Phys.: Conf. Ser. {\bf898} 082033}

%section 2 references
%section 2 Tier 0/1

%section 2 Tier 2

%section 2 Experiments
\bibitem{alien}  http://cds.cern.ch/record/1171677?ln=en

\bibitem{jalien} http://cds.cern.ch/record/2026281?ln=en

\bibitem{glideinwms} 
http://iopscience.iop.org/article/10.1088/1742-6596/119/6/062044

\bibitem{htcondor}
Douglas Thain, Todd Tannenbaum, and Miron Livny.
\newblock Distributed computing in practice: the condor experience.
\newblock {\em Concurrency - Practice and Experience}, 17(2-4):323--356, 2005.

\bibitem{dirac} Dirac :  A Tsaregorodtsev and the Dirac Project 2014 J. Phys.: Conf.
Ser. 513 032096
LHCb : LHCb collaboration, A. A. Alves Jr. et al., The LHCb detector at
the LHC, JINST 3 (2008) S08005

%section 3 references
\bibitem{sam}
A~Aimar et al 
%A~Aguado Corman, P~Andrade, S~Belov, J~Delgado Fernandez, B~Garrido Bear, M~Georgiou, E~Karavakis, L~Magnoni, R~Rama Ballesteros, H~Riahi, J~Rodriguez Martinez, P~Saiz, and D~Zolnai.
\newblock Unified monitoring architecture for it and grid services.
\newblock {\em Journal of Physics: Conference Series}, 898(9):092033, 2017.

\bibitem{etf}
Marian Babik.
\newblock Experiments {Test} {Framework} ({ETF}).

\bibitem{perfsonar} 
Andreas Hanemann, Jeff~W. Boote, Eric~L. Boyd, J{\'e}r{\^o}me Durand, Loukik
  Kudarimoti, Roman {\L}apacz, D.~Martin Swany, Szymon Trocha, and Jason
  Zurawski.
\newblock Perfsonar: A service oriented architecture for multi-domain network
  monitoring.
\newblock In Boualem Benatallah, Fabio Casati, and Paolo Traverso, editors,
  {\em Service-Oriented Computing - ICSOC 2005}, pages 241--254, Berlin,
  Heidelberg, 2005. Springer Berlin Heidelberg.

\bibitem{wlcg-NTWG}
S.~McKee, M.~Babik et al 
%S.~Campana, A.~Di Girolamo, T.~Wildish, J.~Closier, S.~Roiser, C.~Grigoras, I.~Vukotic, M.~Salichos, Kaushik De, V.~Garonne, J.A.D. Cruz, A.~Forti, C.J. Walker, D.~Rand, A.~de~Salvo, E.~Mazzoni, I.~Gable, F.~Chollet, L.~Caillat, F.~Schaer, Hsin-Yen Chen, U.~Tigerstedt, G.~Duckeck, B.~Hoeft, A.~Petzold, F.~Lopez, J.~Flix, S.~Stancu, J.~Shade, M.~O'Connor, V.~Kotlyar, and J.~Zurawski.
\newblock Integrating network and transfer metrics to optimize transfer
  efficiency and experiment workflows.
\newblock {\em Journal of Physics: Conference Series}, 664(5):052003, 2015.

\bibitem{psmad}  
perfSONAR Consortium.
\newblock {perfSONAR} {Monitoring and Debugging Dashboard (MADDASH)}.

\bibitem{grafana-ipv6}
CERN~{MONIT} team.
\newblock {CERN} {MONIT} {Grafana} {Dashboard}.

\bibitem{638647551}
A~A Ayllon, M~Salichos, M~K Simon, and O~Keeble.
\newblock Fts3: New data movement service for wlcg.
\newblock {\em J. Phys.: Conf. Ser}, 513(3):032081, 2014.

\bibitem{grafana-FTS}
CERN~{MONIT} team.
\newblock {CERN} {MONIT} {Grafana} {FTS} {Dashboard}.
 
\bibitem{grafana-WLCG-Transfers}
CERN~{MONIT} team.
\newblock {CERN} {MONIT} {Grafana} {WLCG} {Transfers} {Dashboard}.


\bibitem{xrootd-ipv6}
SLAC~{XRootD} team.
\newblock {XRootD System Monitoring Reference}.

%section 4 references
\bibitem{rfc} All Internet Engineering Task Force Requests For Comments (RFC) documents are available
from URLs such as http://www.ietf.org/rfc/rfcNNNN.txt where NNNN is the RFC number, for example {\tt http://www.ietf.org/rfc/rfc2460.txt}

\end{thebibliography}

\end{document}

