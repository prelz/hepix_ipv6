\documentclass[a4paper]{jpconf}
\usepackage{graphicx}
\usepackage{color}
\usepackage{array}
\usepackage{enumerate}

\begin{document}
\title{WLCG and IPv6 - the HEPiX IPv6 working group}

\author{S Campana$^1$, K Chadwick$^2$, G Chen$^3$, J Chudoba$^4$, P Clarke$^5$, M Eli\'a\v s$^4$,
        A Elwell$^1$, S Fayer$^6$, T Finnern$^7$, L Goossens$^1$,
        C Grigoras$^1$, B Hoeft$^8$, D P Kelsey$^9$, T Kouba$^4$, 
        F L\'opez Mu\~noz$^{10}$, E Martelli$^1$, M Mitchell$^{11}$, A Nairz$^1$, 
        K Ohrenberg$^7$, A Pfeiffer$^1$, F Prelz$^{12}$, F Qi$^3$, D Rand$^6$, 
        M Reale$^{13}$, S Rozsa$^{14}$, A Sciab\`a$^1$, R Voicu$^{14}$, 
        C J Walker$^{15}$ and T Wildish$^{16}$}

\address{$^1$ CERN, CH-1211 Gen\`eve 23, Switzerland}
\address{$^2$ Fermi National Accelerator Laboratory, Batavia, Il 60510, U.S.A.}
\address{$^3$ Institute of High Energy Physics, 19B Yuquanlu, Shijingshan District, 100049 Beijing, China} 
\address{$^4$ Institute of Physics, Academy of Sciences of the Czech Republic Na Slovance 2 182 21 Prague 8, Czech Republic}
\address{$^5$ The University of Edinburgh, Mayfield Road, Edinburgh EH9 3JZ, United Kingdom}
\address{$^6$ Imperial College London, South Kensington Campus, London SW7 2AZ, United Kingdom}
\address{$^7$ Deutsches Elektronen-Synchrotron, Notkestra\ss e 85, D-22607 Hamburg, Germany}
\address{$^8$ Karlsruher Institut f\"ur Technologie, Hermann-von-Helmholtz-Platz 1, D-76344 Eggenstein-Leopoldshafen, Germany}
\address{$^9$ STFC Rutherford Appleton Laboratory, Harwell Oxford, Didcot, Oxfordshire OX11 0QX, United Kingdom}
\address{$^{10}$ Port d'Informaci\'o Cient\'ifica, Campus UAB, Edifici D, E-08193 Bellaterra, Spain}
\address{$^{11}$ University of Glasgow, Kelvin Building, University Avenue, Glasgow G12 8QQ, United Kingdom}
\address{$^{12}$ INFN, Sezione di Milano, via G. Celoria 16, I-20133 Milano, Italy}
\address{$^{13}$ Consortium GARR, Via dei Tizii 6, I-00185 Roma, Italy}
\address{$^{14}$ California Institute of Technology, Pasadena, Ca 91125, U.S.A.}
\address{$^{15}$ Queen Mary University of London, Mile End Road, London E1 4NS, United Kingdom}
\address{$^{16}$ Princeton University, Jadwin Hall, Princeton, NJ 08544, U.S.A.}

\ead{david.kelsey@stfc.ac.uk, ipv6@hepix.org}

\begin{abstract}
The HEPiX ({\tt http://www.hepix.org}) IPv6 Working Group has 
been investigating the many issues which feed into the decision on the 
timetable for the use of IPv6 ({\tt http://www.ietf.org/rfc/rfc2460.txt}) 
networking protocols in High Energy Physics (HEP) Computing, in 
particular in the Worldwide LHC Computing Grid (WLCG).
RIPE NCC, the European Regional Internet Registry (RIR), ran 
out of IPv4 addresses in September 2012. The North and South America RIRs 
are expected to run out soon. In recent months it has become more clear 
that some WLCG sites, including CERN, are running short of IPv4 address space, 
now without the possibility of applying for more. This has increased the 
urgency for the switch-on of dual-stack IPv4/IPv6 on all outward facing WLCG 
services to allow for the eventual support of IPv6-only clients. The 
activities of the group include the analysis and testing of the readiness 
for IPv6 and the performance of many required components, including the 
applications, middleware, management and monitoring tools essential for 
HEP computing. Many WLCG Tier 1/2 sites are participants in the 
group's distributed IPv6 testbed and the major Large Hadron Collider (LHC)
experiment collaborations 
are engaged in the testing.  We are constructing a group 
web/wiki which will contain useful information on the IPv6 
readiness of the various software components and a knowledge base
({\tt http://hepix-ipv6.web.cern.ch/knowledge-base}).
This paper describes the work done by the working group and its future plans.
\end{abstract}

\section{Introduction}
The world's Regional Internet Registries are rapidly running out of available IPv4 addresses and the 
general slow transition to IPv6 continues. The Worldwide Large Hadron Collider Grid (WLCG) and the LHC experiments 
may soon have access to worker nodes or virtual machines possessing only an IPv6-routable address. The HEPiX
IPv6 Working Group [Ref 1] has been investigating the many issues feeding into the move to the use of IPv6 in HEP and WLCG.
The group's paper at CHEP2013 [Ref 2] described the aims of the group and the testing of dual-stack IPv6/IPv4 
services that had been completed at that point. In the last 18 months the group has worked more closely with the
4 major LHC experiments and identified the main use case for the support of IPv6-only clients on WLCG. The groups
activities, including testing of dual-stack data storage services, during the last 18 months are presented in this 
paper together with its future plans.

%\section{IPv6: the general problem}
\section{IPv6: the general problem}

The intention of the designers of the IPv6 protocol was to make it full of appealing features, in order to push its adoption widely and quickly. The IPv6 specifications (RFC 1883) were set back in the 1995, when the Internet community realized that the classfull allocation policies of the time were causing a quick depletion of the address space. 
\par
Unfortunately for IPv6, the IPv4 problem was quickly fixed with the adoption of the classless allocations (CIDR, RFC 1519) and by the invention of Address and Port translation techniques (NAT, RFC 1631). These events, together with the fact that the IPv6 advantages were far less appealing than the cost of deploying it, put the protocol in a limbo where it stayed for almost twenty years, until IPv4 addresses became scarce again.
\par
At the end of the first decade of the 21st century, Regional Internet Registries started warning the Internet community that IPv4 addresses were soon be exhausted and urged everyone to adopt IPv6. IPv6 was quite quickly deployed on the Internet backbones, but not where it would have brought the most of its benefits, at the client and content side. 
Plagued by the chicken and the egg problem (no users if no content, no content if no users), in 2012 finally some of the 
biggest content providers made the bold move to make their services available over IPv6. 
One year later, the IPv6 global traffic is gradually increasing but still counts as a very small fraction of the 
total Internet traffic.

\section{IPv6 at CERN}
Many academic institutions, which joined the Internet when it was in a very early stage, are still enjoying the large allocations that were given in those days; thus they are lacking of any urge to move to IPv6.
\par 
This was the situation at CERN till 2011, when Server Virtualization started being used. The virtualization technique proved to be very effective and its adoption at CERN has grown exponentially. In 2012 , when the plan for the services to be run in the upcoming remote data centre in Wigner (Budapest, HU) was finalized, it became clear that something like 250,000 public IP addresses would be needed in the near future. At the same time, RIPE, the European Internet Registry, was announcing the adoption of a new conservative allocation policy that would grant no more than 1024 IPv4 addresses to any requester. 
It could have been an impasse for the IT deployment plans, but luckily CERN had been testing with IPv6 since 1998 and in 2011 the management of the IT department approved the project to deploy IPv6 in the CERN campus and datacentres, when its need had yet to be proven.
\par
For large enterprises like CERN, deploying IPv6 is not as simple as configuring a dozen of routers to be dual stack. The Network Management System and the Network database had to be made IPv6 aware, all the IPv6 information generated, all the basic network services configured (DNS, NTP, DHCPv6..). At the same time the network security had to be kept at the same level as always.
After two years, the deployment is almost completed. Right in time to tackle the IPv4 exhaustion problem that most likely will hit CERN in 2015 when the Wigner datacentre will reach its full capacity. 
Many applications still cannot make use of IPv6, thus is very premature to deploy IPv6 only Virtual Servers. The strategy will be to deploy an hybrid solution where servers get a private IPv4 address and a public IPv6 one. The private IPv4 address will allow legacy applications to work within the CERN domain,  while the public IPv6 address will allow world-wide reachability.
\par
Hopefully the availability of LHC data over IPv6 will push IPv6 adoption in the large WLCG community.


\section{The HEPiX IPv6 working group}
The HEPiX forum brings together worldwide IT staff, including system administrators, system engineers, and managers from the High Energy Physics and Nuclear Physics laboratories and institutes, to foster a learning and sharing experience between sites facing scientific computing and data challenges. At its semi-annual meetings, HEPiX had been considering the issue of migration to IPv6 for a number of years. In 2011 a survey of HEP sites around the world was made asking about their plans for the deployment of IPv6. While it was very clear that there was no requirement for an urgent move to IPv6 a good number of sites were planning such a deployment and a few, particularly CERN, reported a foreseen lack of IPv4 address space in the not too distant future.

It was realised that any decision to deploy IPv6 on the WLCG infrastructure would involve much testing and planning and the decision was therefore taken to start a dedicated working group to investigate the issues. The HEPiX IPv6 working group was formed in 2011 with the following mandate.

Phase 1 of the work was to consider whether and how IPv6 should be deployed in HEP (especially for WLCG). This involved:
	•	A Readiness and Gap analysis
	•	The need to include all relevant HEP applications, middleware, security issues, system management and monitoring tools, and end-to-end network monitoring
	•	Running a distributed HEPiX IPv6 testbed to explore all of the above issues
	•	An initial report at the end of 2011
	•	
Following that initial report it was agreed that the work should continue and that eventually phase 2 of the work should include:
	•	The proposal of a timetable and an analysis of the resources required for the deployment of IPv6 on WLCG
	•	The production of an implementation plan including advice to HEP sites on deployment

Since then the group has been working on the mandate and meeting regularly with quarterly face to face meetings at CERN and monthly video/phone meetings to review progress. 

Full details of the meetings are available at {\tt http://indico.cern.ch/categoryDisplay.py?categId=3538}




\section{The IPv6 testbed}
The impact of IPv6 is not limited to the transport layer 
but introduces the need for choice and preference in name-to-address 
resolution, implies multi-homing of all network endpoints (possibly on 
multiple protocol versions) and requires opaque handling of address 
information. This broadens the scope of code changes needed to add
IPv6 support to existing code and adds to the complexity of testing:
continued operation on IPv4 on dual-stack hosts, then preference of IPv6 and 
options to control it for all network bindings and connections
need to be verified with adequate code coverage.
\par
At opposite ends of the spectrum of practical testing options are
testing of individual, isolated components and services and the analysis of 
integrated services on dual-stack nodes. Both approaches are incomplete:
\begin{itemize}
\item[-] testing of isolated components misses the interaction with other
services at the OS level and usually requires services to be configured
differently than for production;
\item[-] testing of production-ready, integrated nodes may just be 
accidentally focusing on normal operation and bring insufficient 
code and functionality coverage.
\end{itemize}
This calls for a complementary approach,
where individual services are deployed and tested within the scale of
available dedicated resources and, once sufficient confidence and knowledge
of their level and mode of IPv6 support is built, are watched in the context
of a production node with dual-stack network and dual IPv4/IPv6 public address
resolution.
\par
Desirable characteristics of a dedicated testbed for single-service testing are:
\begin{itemize}
\item Geographical spread covering all ranges of realistic network latencies
and as many network providers as possible.
\item Uniform authentication/authorization scheme to factor out AA issues.
\item Uniform OS installation, to factor out any issue with the custom 
configuration needed to test isolated services and for easier
deployment of new services.
\end{itemize}
The current list of active testbed nodes can be found at the following URL:\\
{\tt http://hepix-ipv6.web.cern.ch/testbed-nodes}\\
While we have at the time of writing a reasonable
9-site/6-NREN coverage of Europe, the only non-european sites in the testbed
are IHEP Beijing in China and Fermilab in the US. More testbed sites are both needed and welcome to join 
to achieve a better match of our stated testbed goals.
\par
As for OS distributions, testbed nodes are mostly installed with 
Scientific Linux (CERN) version 5,
to replicate production conditions at LCG Tier-X centres. A few testbed nodes
have RHEL 6 derivatives installed: this allowed us to discover 
(and document in our knowledge base, {\tt http://hepix-ipv6.web.cern.ch/knowledge-base})
that, rather unexpectedly, {\tt libc} on RH6 causes unspecified 
protocol sockets to be bound on IPv4 only, instead of dual-stack 
as it used to be.
\par
To achieve the simplest possible authentication scheme, a custom Globus
Security Infrastructure (GSI)
plug-in maps all members of our test VO ({\tt ipv6.hepix.org}) to one
local account, logging any access. 
\par
The first service we deployed for standalone testing through the testbed was
GridFTP (or, more accurately, {\tt gsiftp}). 
This was not only because of GridFTP's basic role in WLCG data
transfer, but also because the FTP protocol (the GSI extensions don't 
affect but also suffer from this issue) is a paradigmatic example of 
how non-trivial IPv6 support can be. The original FTP specifications 
(RFC765/RFC959) used the quad-byte notation for IP addresses 
{\em in the syntax of the FTP protocol commands} {\tt PORT} and {\tt PASV}.
This required the introduction, with RFC 2428 (September 1998) of
``extended'' versions of the same commands, supporting different address
families, and IPv6 in particular. We found on our testbed that support for 
the 'extended' command forms (and thus {\em implicitly} for IPv6)
is missing from certain FTP client implementations. Retrofitting the
clients with these commands is definitely more than a simple change in
the transport layer and serves as an example of how ramified the
introduction of 'IPv6 support' can be.
\par
Building on this mesh of GridFTP servers, both continuous direct point-to-point 
file transfer tests and tests of the File Transfer Service 
(FTS), were successfully carried out. SRM endpoints
were also added, as described in detail in the next section.


\section{IPv6 testing and results}
%\subsection{Testing testing...}
This is where the testing results go

some numbers for now: Transfers running since March, some 2.6 million transfers, 87\% success
rate, over 2 PB of data so far. This is approximately 7\% of the rate CMS achieves globally.

% Points to make:
We used the PhEDEx LifeCycle agent \cite{LifeCycle} to drive transfers between pairs of sites, using gridftp with the IPv6 connectivity flags. Filesizes were checked at the destination, and any failures recorded. Files were transferred in both directions between each site pair.

Initially, we simply tested connectivity and basic functionality. We also tested under specific conditions, e.g. to compare throughput and error rates with IPv6 vs. IPv4 connectivity. This was useful for debugging issues with firewalls etc.

Since March 2013 the transfer testbed has been running continuously, with more sites joining over time. Finally we have 11 sites transferring 1 GB files between each other. With this many sites, we have had to introduce a delay between successive transfers, to reduce load on the servers.

To date, we have transferred over 2 PB of data between the 11 sites over the 6 months since the testbed started continuous operations. This is 7\% of the rate that CMS achieve in daily operations, so not an insignificant amount. The overall success rate for transfers is 87\%, which is very high considering that the testbed was operated at-risk, with errors only detected when someone decided to look for them.

% 2.6 M transfers, 87\% success, over 2 PB transferred, which is ~ 7\% of global CMS rate
See figure \ref{fig:full-mesh} for the full mesh.

% discuss results of figure full-mesh

% PhEDEx transfers. using private PhEDEx instance and nodes. DPM and FTS3. Transfers throttled to avoid overload. PhEDEx now shown to run with IPv6 endpoints.
% Simon/Duncan to describe this bit better

% To create this graphic:
% 1) save your image as a 1024x1024 png/gif/bmp
% 2) convert to pdf (install ImageMagick, then 'convert FileIn.png FileOut.pdf')
% N.B. if the input and output files have the same base name, LaTeX will prefer to take the png over the pdf,
% which is probably not what you want. Make sure the files have different names!
\begin{figure}[htp]
\centering
\includegraphics{full-mesh}
\caption{Transfer performance for the IPv6 testbed continuous transfers. A 1 GB file is transferred between each pair of sites, then deleted, then transferred again, continuously. The plots show the distribution of transfer duration times per site pair. The source site is named in the row, the destination site is named in the column. So the top-right plot shows transfers from Caltech to PIC, the bottom-left shows transfers fromPIC to Caltech. The x-axis is in seconds, from 0 to 500 for each plot. The number inset in each plot shows the approximate number of transfers between that site pair in that direction.}\label{fig:full-mesh}
\end{figure}

\begin{figure}[htp]
\centering
\includegraphics{phedex-transfer-volume}
\caption{Cumulative data-transfer between Imperial College and Glasgow using PhEDEx on the IPv6 testbed.}\label{fig:phedex-transfer-volume}
\end{figure}



\section{Software and tools survey}
For a successful transition to the use of IPv6 it is necessary to do a full survey of all important applications, middleware and operational tools. We decided to focus on the IPv6 readiness of the important WLCG outward-facing services and all essential applications and management and monitoring tools. We have created an online database of IPv6-readiness where for each software component considered we can store its known state of readiness and details of any testing performed by us or others.

IPv6-readiness is not a simple yes/no question. There are various stages of readiness to be addressed.
Does the service break/slow down when used with IPv4 on a dual-stack host with IPv6 enabled?
Will the service try using (connecting/binding to) an IPv6 address (AAAA record), when available from DNS?
Will the service prefer IPv6 addresses from DNS, when preferred at the host level?
Does this need to be configured and how?
Can the service be persuaded to fall back on IPv4 if needed?
In many ways the most important question is the first one. As long as a service deployed on a dual-stack host behaves properly for IPv4 then we are safe to recommend such a deployment on the WLCG production infrastructure.
The current state of our survey may be seen at
{\tt http://hepix-ipv6.web.cern.ch/wlcg-applications}.

At the time of 
writing, there are still many packages which need further investigation and testing. Software known not to be ready for IPv6 at this time includes OpenAFS servers and clients, all but the latest release of dCache and many batch systems. Full details of all such problems and investigations will be recorded in our online database.




%\section{Collaboration with other groups}
%\input{section-7-collaboration-with-other-groups.tex}

\section{Outlook and future plans}
The IPv6 working group has made good progress but has to date still only tested a small fraction of the many 
software components required by WLCG. There are still many more tests and 
assessments to be made and advice on dual-stack deployment to be formulated before 
our work can in any way be defined as complete.

Testing will continue in three areas. Firstly we will continue the mesh of data transfer tests between
the testbed sites, expanding to include new types of storage element and allowing for the ongoing
assessment of reliability. Some testbed sites are now working on the deployment of larger scale IPv6 testbeds
to allow the testing of IPv6-readiness in a more realistic production environment. Finally, we will encourage 
more Tier 2 sites to repeat the
tests at Imperial College (see section 6.4) by enabling dual-stack services on some or all of their production
services. Across all of these different scenarios we will gradually work through the list of testing scenarios presented
in section 6.1 together with representatives of the experiments.  During all of this testing we
will continue to update our online database with the details of IPv6 readiness.

At the time of our status report to HEPiX and the WLCG Management Board in 2012 we concluded that
the support of IPv6-only clients on WLCG was unlikely to be possible before January 2014. At the time of writing this is still true and indeed certain to take much longer. Not only does the working group still have many tests to perform
but all of the many Tier 1 and Tier 2 sites need to complete the deployment of an IPv6 infrastructure at their site. This includes
the necessary revision of local procedures and management tools and the provision of adequate training for network,
system operations and security staff. Once we have achieved this we can propose a more general deployment of 
production-level dual stack services thereby allowing for the support of IPv6-only clients on WLCG.


\par
\begin{thebibliography}{1}
%
% and use \bibitem to create references.
%
% Format for Journal Reference
%Journal Author, Journal \textbf{Volume}, page numbers (year)
% Format for books
%\bibitem{RefB}
%Book Author, \textit{Book title} (Publisher, place, year) page numbers
% etc


%section 1 references
\bibitem{ipv6wg} The HEPiX IPv6 Working Group web site is to be found at {\tt http://hepix-ipv6.web.cern.ch}

\bibitem{ipv6chep2016} 
Babik M et al 2016 Deployment of IPv6-only CPU resources at WLCG sites {\it J. Phys.: Conf. Ser. {\bf898} 082033}

%section 2 references
%section 2 Tier 0/1

%section 2 Tier 2

%section 2 Experiments
\bibitem{alien}  http://cds.cern.ch/record/1171677?ln=en

\bibitem{jalien} http://cds.cern.ch/record/2026281?ln=en

\bibitem{glideinwms} 
http://iopscience.iop.org/article/10.1088/1742-6596/119/6/062044

\bibitem{htcondor}
Douglas Thain, Todd Tannenbaum, and Miron Livny.
\newblock Distributed computing in practice: the condor experience.
\newblock {\em Concurrency - Practice and Experience}, 17(2-4):323--356, 2005.

\bibitem{dirac} Dirac :  A Tsaregorodtsev and the Dirac Project 2014 J. Phys.: Conf.
Ser. 513 032096
LHCb : LHCb collaboration, A. A. Alves Jr. et al., The LHCb detector at
the LHC, JINST 3 (2008) S08005

%section 3 references
\bibitem{sam}
A~Aimar et al 
%A~Aguado Corman, P~Andrade, S~Belov, J~Delgado Fernandez, B~Garrido Bear, M~Georgiou, E~Karavakis, L~Magnoni, R~Rama Ballesteros, H~Riahi, J~Rodriguez Martinez, P~Saiz, and D~Zolnai.
\newblock Unified monitoring architecture for it and grid services.
\newblock {\em Journal of Physics: Conference Series}, 898(9):092033, 2017.

\bibitem{etf}
Marian Babik.
\newblock Experiments {Test} {Framework} ({ETF}).

\bibitem{perfsonar} 
Andreas Hanemann, Jeff~W. Boote, Eric~L. Boyd, J{\'e}r{\^o}me Durand, Loukik
  Kudarimoti, Roman {\L}apacz, D.~Martin Swany, Szymon Trocha, and Jason
  Zurawski.
\newblock Perfsonar: A service oriented architecture for multi-domain network
  monitoring.
\newblock In Boualem Benatallah, Fabio Casati, and Paolo Traverso, editors,
  {\em Service-Oriented Computing - ICSOC 2005}, pages 241--254, Berlin,
  Heidelberg, 2005. Springer Berlin Heidelberg.

\bibitem{wlcg-NTWG}
S.~McKee, M.~Babik et al 
%S.~Campana, A.~Di Girolamo, T.~Wildish, J.~Closier, S.~Roiser, C.~Grigoras, I.~Vukotic, M.~Salichos, Kaushik De, V.~Garonne, J.A.D. Cruz, A.~Forti, C.J. Walker, D.~Rand, A.~de~Salvo, E.~Mazzoni, I.~Gable, F.~Chollet, L.~Caillat, F.~Schaer, Hsin-Yen Chen, U.~Tigerstedt, G.~Duckeck, B.~Hoeft, A.~Petzold, F.~Lopez, J.~Flix, S.~Stancu, J.~Shade, M.~O'Connor, V.~Kotlyar, and J.~Zurawski.
\newblock Integrating network and transfer metrics to optimize transfer
  efficiency and experiment workflows.
\newblock {\em Journal of Physics: Conference Series}, 664(5):052003, 2015.

\bibitem{psmad}  
perfSONAR Consortium.
\newblock {perfSONAR} {Monitoring and Debugging Dashboard (MADDASH)}.

\bibitem{grafana-ipv6}
CERN~{MONIT} team.
\newblock {CERN} {MONIT} {Grafana} {Dashboard}.

\bibitem{638647551}
A~A Ayllon, M~Salichos, M~K Simon, and O~Keeble.
\newblock Fts3: New data movement service for wlcg.
\newblock {\em J. Phys.: Conf. Ser}, 513(3):032081, 2014.

\bibitem{grafana-FTS}
CERN~{MONIT} team.
\newblock {CERN} {MONIT} {Grafana} {FTS} {Dashboard}.
 
\bibitem{grafana-WLCG-Transfers}
CERN~{MONIT} team.
\newblock {CERN} {MONIT} {Grafana} {WLCG} {Transfers} {Dashboard}.


\bibitem{xrootd-ipv6}
SLAC~{XRootD} team.
\newblock {XRootD System Monitoring Reference}.

%section 4 references
\bibitem{rfc} All Internet Engineering Task Force Requests For Comments (RFC) documents are available
from URLs such as http://www.ietf.org/rfc/rfcNNNN.txt where NNNN is the RFC number, for example {\tt http://www.ietf.org/rfc/rfc2460.txt}

\end{thebibliography}

\end{document}

